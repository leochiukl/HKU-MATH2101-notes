\section{Linear Transformations}
\label{sect:linear-transformations}
\begin{enumerate}
\item We have mentioned in
\Cref{subsect:matrix-vec-product,subsect:matrix-mult} that matrix-vector
product and matrix multiplication are actually related to the concept of
\emph{linear transformation}. As we have mentioned at the very beginning of
this notes, linear transformation is another central concept in linear algebra,
apart from vector space. So we will investigate linear transformations in
\Cref{sect:linear-transformations}.
\end{enumerate}
\subsection{Linear Transformations}
\begin{enumerate}
\item Given two vector spaces, we can define many functions by treating one of
them as domain as another as codomain. However, not all such functions are said
to be \emph{linear transformations}. To qualify as a linear transformation, we
impose an additional constraint: The operations on vector spaces, namely vector
addition and scalar multiplication, should be preserved.

\item Let \(V\) and \(W\) be vector spaces. A function \(T:V\to W\) is a
\defn{linear transformation} from \(V\) to \(W\) if for any
\(\vect{u},\vect{v}\in V\) and any scalar \(c\in\R\),
\begin{enumerate}
\item (preserving vector addition) \(T(\vect{u}+\vect{v})=T(\vect{u})+T(\vect{v})\);
\item (preserving scalar multiplication) \(T(c\vect{v})=cT(\vect{v})\).
\end{enumerate}
Examples:
\begin{itemize}
\item A function \(T:\R^2\to\R^2\) defined by \(T\qty(\mqty[x\\ y])=3\mqty[x\\ y]\)
is a linear transformation.

\begin{pf}
Exercise.
\end{pf}
\item A function \(T:\R^n\to\R\) defined by \(T\qty(\mqty[x_1\\ \vdots\\ x_n])=x_1+\dotsb+x_n\)
is a linear transformation.

\begin{pf}
Exercise.
\end{pf}
\item A function \(T:\R^3\to\R^2\) defined by \(T\qty(\mqty[x_1\\ x_2\\ x_3])=\mqty[x_1+x_2\\ x_2-2x_3]\)
is a linear transformation.

\begin{pf}
Fix any \(\vect{u}=\mqty[x_1\\ x_2\\ x_3]\) and \(\vect{v}=\mqty[y_1\\ y_2\\
y_3]\) in \(\R^3\). Then,
\begin{align*}
T(\vect{u}+\vect{v})
&=T\qty(\mqty[\blc{x_1+y_1}\\ \gc{x_2+y_2}\\ \orc{x_3+y_3}]) \\
&=\mqty[\blc{x_1+y_1}+\gc{x_2+y_2}\\ \gc{x_2+y_2}-2(\orc{x_3+y_3})] \\
&=\mqty[x_1+x_2\\ x_2-2x_3]+\mqty[y_1+y_2\\ y_2-2y_3] \\
&=T\qty(\mqty[x_1\\ x_2\\ x_3])+T\qty(\mqty[y_1\\ y_2\\ y_3]) \\
&=T(\vect{u})+T(\vect{v}).
\end{align*}
Next, for any \(c\in\R\), we have
\[
T(c\vect{v})
=T\qty(c\mqty[x_1\\ x_2\\ x_3])
=T\qty(\mqty[cx_1\\ cx_2\\ cx_3])
=\mqty[cx_1+cx_2\\ cx_2-2cx_3]
=c\mqty[x_1+x_2\\ x_2-2x_3]
=cT\qty(\mqty[x_1\\ x_2\\ x_3])
=cT(\vect{v}).
\]
\end{pf}
\end{itemize}
\item \label{it:matx-mult-lt} To connect the concepts of matrix-vector product and linear
transformation, consider the following. Let \(A\) be an \(m\times n\) matrix,
and \(L_A:\R^n\to\R^m\) be defined by \(L_A(\vect{v})=A\vect{v}\) for any \(\vect{v}\in\R^n\).
Then \(L_A\) is a linear transformation.

\begin{pf}
Fix any \(\vect{u},\vect{v}\in\R^n\). Firstly, we have
\(L_A(\vect{u}+\vect{v})=A(\vect{u}+\vect{v})=A\vect{u}+A\vect{v}=L_A(\vect{u})+L_A(\vect{v})\).
Next, for any \(c\in\R\), we have
\(L_A(c\vect{v})=A(c\vect{v})=c(A\vect{v})=cL_A(\vect{v})\).
\end{pf}

\item Two special kinds of linear transformations are as follows. Let \(V\) and
\(W\) be vector spaces.
\begin{itemize}
\item The identity function \(\id_{V}:V\to V\), defined by
\(\id_{V}(\vect{v})=\vect{v}\) for any \(\vect{v}\in V\), is a linear
transformation. It is called the \defn{identity transformation}.

\begin{pf}
To prove that it is a linear transformation, first fix any
\(\vect{u},\vect{v}\in V\). Then,
\[
\id_{V}(\vect{u}+\vect{v})=\vect{u}+\vect{v}=\id_{V}(\vect{u})+\id_{V}(\vect{v})
\]
and for any \(c\in\R\), we have
\[
\id_{V}(c\vect{v})=c\vect{v}=c\id_{V}(\vect{v}).
\]
\end{pf}

\item The zero function \(T_0:V\to W\), defined by
\(T_0(\vect{v})=\vect{0}\)\footnote{Note that \(\vect{0}\) is the zero vector
\underline{in \(W\)}.} for any \(\vect{v}\in V\), is a linear
transformation. It is called the \defn{zero transformation}.

\begin{pf}
Again first fix any \(\vect{u},\vect{v}\in V\). Then,
\[
T_0(\vect{u}+\vect{v})=\vect{0}=\vect{0}+\vect{0}=T_0(\vect{u})+T_0(\vect{v})
\]
and for any \(c\in\R\), we have
\[
T_0(c\vect{v})=\vect{0}=c\vect{0}=cT_0(\vect{v}).
\]
\end{pf}
\end{itemize}
For the identity transformation, if \(V=\R^n\), then it is indeed a special
case of the linear transformation in \labelcref{it:matx-mult-lt} with \(A=I_n\).
For the zero transformation, if \(V=\R^n\) and \(W=\R^m\), then it is a special
case of the linear transformation in \labelcref{it:matx-mult-lt} with
\(A=O_{m\times n}\). Note however that the identity and zero transformations
can be defined for arbitrary vector spaces, not limited to \(\R^n\).

\item The following result suggests that linear transformation also preserves
\emph{zero} and \emph{linear combinations}.
\begin{proposition}
\label{prp:lt-zero-to-zero-preserv-lincomb}
Let \(T:V\to W\) be a linear transformation. Then:
\begin{enumerate}
\item \(T(\vect{0})=\vect{0}\).
\begin{note}
The \(\vect{0}\) in the LHS denotes the zero vector \underline{in \(V\)}, while
the \(\vect{0}\) in the RHS denotes the zero vector \underline{in \(W\)}.
\end{note}
\item Let \(n\) be any positive integer. For any \(a_1,\dotsc,a_n\in\R\) and
any \(\vect{v}_1,\dotsc,\vect{v}_n\in V\),
\[
T(a_1\vect{v}_1+\dotsb+a_n\vect{v}_n)=a_1T(\vect{v}_1)+\dotsb+a_nT(\vect{v}_n).
\]
\end{enumerate}
\end{proposition}
\begin{pf}
\begin{enumerate}
\item Fix any \(\vect{v}\in V\) and note that
\[
T(\vect{0})=T(0\vect{v})=0T(\vect{v})=\vect{0},
\]
by \Cref{prp:vector-space-prop}.

\item We prove by induction. The case \(n=1\) follows immediately from the
definition of linear transformation (preservation of scalar multiplication).
Now assume that the case \(n=k\) for a \(k\in\N\), i.e.,
\[
T(a_1\vect{v}_1+\dotsb+a_k\vect{v}_k)=a_1T(\vect{v}_1)+\dotsb+a_kT(\vect{v}_k).
\]
Consider
\begin{align*}
T(\blc{a_1\vect{v}_1+\dotsb+a_k\vect{v}_k}+a_{k+1}\vect{v}_{k+1})
&=T(\blc{a_1\vect{v}_1+\dotsb+a_k\vect{v}_k})+T(a_{k+1}\vect{v}_{k+1}) \\
&=a_1T(\vect{v}_1)+\dotsb+a_kT(\vect{v}_k)+T(a_{k+1}\vect{v}_{k+1}) \\
&=a_1T(\vect{v}_1)+\dotsb+a_kT(\vect{v}_k)+a_{k+1}T(\vect{v}_{k+1}),
\end{align*}
so the case \(n=k+1\) holds, hence the result follows by induction.
\end{enumerate}
\end{pf}

\item More advanced examples of linear transformations:
\begin{itemize}
\item Let \(V\) be the vector space of differentiable functions from \(\R\) to
\(\R\). (It can be proved that \(V\) is a vector subspace of the vector space
of all functions from \(\R\) to \(\R\).) Define a function \(T:V\to V\) by \(T(f)=f'\)
for any \(f\in V\). Then \(T\) is a linear transformation.

\begin{pf}
Fix any \(f_1,f_2\in V\). Firstly, we have
\[
T(f_1+f_2)=(f_1+f_2)'=f_1'+f_2'=T(f_1)+T(f_2).
\]
Next, for any \(c\in\R\), we have
\[
T(cf_1)=(cf_1)'=cf_1'=cT(f_1).
\]
\end{pf}
\item Let \(V\) be the vector space of all \(m\times n\) matrices, \(P\) be an
\(m\times m\) matrix, and \(Q\) be an \(n\times n\) matrix. Define a function
\(T:V\to V\) by \(T(A)=PAQ\) for any \(A\in V\). Then \(T\) is a linear
transformation.

\begin{pf}
Fix any \(A,B\in V\). Firstly, we have
\[
T(A+B)=P(A+B)Q=P(AQ+BQ)=PAQ+PBQ=T(A)+T(B).
\]
Next, for any \(c\in\R\), we have
\[
T(cA)+P(cA)Q=cPAQ=cT(A).
\]
\end{pf}
\end{itemize}
\item The following is a \underline{non}-example of linear transformation. Let
\(V\) be the vector space of \(n\times n\) matrices. Define \(T:V\to V\) by
\(T(A)=A^{2}\). Then \(T\) is \emph{not} a linear transformation since
\(T(2I_n)=(2I_n)^{2}=4I_n^{2}=4T(I_n)\ne 2T(I_n)\).

\item To obtain more examples of linear transformations, we can utilize the
following result to construct linear transformations from \emph{bases}.

\begin{theorem}
\label{thm:construct-lt-from-basis}
Let \(V\) and \(W\) be vector spaces. Let \(\{\vect{v}_1,\dotsc,\vect{v}_n\}\)
be a basis for \(V\) and let \(\vect{w}_1,\dotsc,\vect{w}_n\in W\). Then there
exists a unique linear transformation \(T:V\to W\) such that
\(T(\vect{v}_i)=\vect{w}_i\) for every \(i=1,\dotsc,n\).
\end{theorem}
\begin{note}
This result does \underline{not} say that \(\{\vect{w}_1,\dotsc,\vect{w}_n\}\)
is a basis for \(W\).
\end{note}

\begin{pf}
\underline{Existence}: Fix any \(\vect{v}\in V\). By
\Cref{thm:basis-unique-lin-comb}, the vector \(\vect{v}\) can be written as
\(\vc{a_1}\vect{v}_1+\dotsb+\vc{a_n}\vect{v}_n\) for some unique scalars
\(\vc{a_1,\dotsc,a_n}\in\R\). Then define \(T:V\to W\) by
\[
T(\vect{v})=\vc{a_1}\vect{w}_1+\dotsb+\vc{a_n}\vect{w}_n.
\]
Note that for every \(i=1,\dotsc,n\), we have \(T(\vect{v}_i)=\vect{w}_i\)
since the corresponding unique scalars are \(a_i=1\) and \(a_j=0\) for any
\(j\ne i\). We also claim that \(T\) is a linear transformation.

\begin{pf}
\underline{Addition}: Fix any \(\vect{v},\vect{v}'\in V\). By
\Cref{thm:basis-unique-lin-comb}, we can write
\[
\vect{v}=a_1\vect{v}_1+\dotsb+a_n\vect{v}_n
\qqtext{and}
\vect{v}'=a_1'\vect{v}_1+\dotsb+a_n'\vect{v}_n
\]
for some unique scalars \(a_1,\dotsc,a_n,a_1',\dotsc,a_n'\in\R\). Note that
\[
\vect{v}+\vect{v}'=\vc{(a_1+a_1')}\vect{v}_1+\dotsb+\vc{(a_n+a_n')}\vect{v}_n,
\]
thus
\[
T(\vect{v}+\vect{v}')=\vc{(a_1+a_1')}\vect{w}_1+\dotsb+\vc{(a_n+a_n')}\vect{w}_n
=(a_1\vect{w}_1+\dotsb+a_n\vect{w}_n)+(a_1'\vect{w}_1+\dotsb+a_n'\vect{w}_n)
=T(\vect{v})+T(\vect{v}').
\]

\underline{Scalar multiplication}: Fix any \(\vect{v}\in V\) and \(c\in\R\). By 
\Cref{thm:basis-unique-lin-comb}, write
\(\vect{v}=a_1\vect{v}_1+\dotsb+a_n\vect{v}_n\) for some unique scalars
\(a_1,\dotsc,a_n\in\R\). Then we have
\(c\vect{v}=(ca_1)\vect{v}_1+\dotsb+(ca_n)\vect{v}_n\), thus
\[
T(c\vect{v})=(ca_1)\vect{w}_1+\dotsb+(ca_n)\vect{w}_n
=c(a_1\vect{w}_1+\dotsb+a_n\vect{w}_n)
=cT(\vect{v}).
\]
\end{pf}

\underline{Uniqueness}: Let \(T':V\to W\) be another linear transformation
satisfying \(T'(\vect{v}_i)=\vect{w}_i\) for every \(i=1,\dotsc,n\). Fix any
\(\vect{v}\in V\). Then there are unique scalars \(a_1,\dotsc,a_n\) such that
\(\vect{v}=a_1\vect{v}_1+\dotsb+a_n\vect{v}_n\). Hence,
\begin{align*}
T'(\vect{v})
&=T'(a_1\vect{v}_1+\dotsb+a_n\vect{v}_n) \\
&=a_1T'(\vect{v}_1)+\dotsb+a_nT'(\vect{v}_n) \\
&=a_1\vect{w}_1+\dotsb+a_n\vect{w}_n \\
&=a_1T(\vect{v}_1)+\dotsb+a_nT(\vect{v}_n) \\
&=T(a_1\vect{v}_1+\dotsb+a_n\vect{v}_n) \\
&=T(\vect{v}).
\end{align*}
Since this holds for arbitrary \(\vect{v}\in V\), we conclude that \(T'=T\).
\end{pf}
\end{enumerate}
\subsection{Null Spaces and Ranges}
\begin{enumerate}
\item Given an \(m\times n\) matrix, we can obtain its corresponding null space
and column space, as suggested in \Cref{subsect:col-sp,subsect:null-sp}. We
can do similar things for a linear transformation, and consider its \emph{null
space} and \emph{range}.

\item Let \(T:V\to W\) be a linear transformation. The \defn{null space} of
\(T\), denoted by \(\nul{T}\), is given by \(\{\vect{v}\in V:T(\vect{v})=\vect{0}\}\).
\begin{note}
Here \(\vect{0}\) denotes the zero vector in \(W\).
\end{note}

The \defn{range} of \(T\), denoted by \(\ran T\), is given by
\(\ran{T}=\{T(\vect{v})\in W:\vect{v}\in V\}\). \begin{note} This coincides
with the usual definition of range for a function.
\end{note}

\item Examples:
\begin{itemize}
\item Consider the zero transformation \(T_0:V\to W\), defined by
\(T_0(\vect{v})=\vect{0}\) for any \(\vect{v}\in V\). Then, \(\nul{T_0}=V\) and
\(\ran{T_0}=\{\vect{0}\}\).
\item Consider the identity transformation \(\id_{V}:V\to V\), defined by
\(\id_{V}(\vect{v})=\vect{v}\) for any \(\vect{v}\in V\). Then,
\(\nul{\id_{V}}=\{\vect{0}\}\) and \(\ran{\id_{V}}=V\).
\item Let \(T:\R^{2n}\to\R^n\) be defined by
\[
T(\vect{v})=\mqty[I_n&O_{n\times n}\\ O_{n\times n}&O_{n\times n}]\vect{v}
\]
for any \(\vect{v}\in\R^{2n}\). Then,
\[
\nul{T}=\qty{\mqty[0\\ \vdots\\0\\ x_{n+1}\\ \vdots\\ x_{2n}]:x_{n+1},\dotsc,x_{2n}\in\R}
\]
and
\[
\ran{T}=\qty{\mqty[x_1\\ \vdots\\x_n\\ 0\\ \vdots\\ 0]:x_{1},\dotsc,x_{n}\in\R}.
\]
\end{itemize}
\item \label{it:null-ran-relate-matx-sp}
Let \(A\) be an \(m\times n\) matrix and let \(L_A:\R^{n}\to\R^{m}\) be the
associated linear transformation, i.e., the one defined by
\(L_A(\vect{v})=A\vect{v}\) for any \(\vect{v}\in\R^n\).

The null space of \(L_A\) is
\[
\nul{L_A}=\{\vect{v}\in \R^n:L_A(\vect{v})=\vect{0}\}
=\{\vect{v}\in\R^n: A\vect{v}=\vect{0}\}
=\nul{A}
\]
as expected. On the other hand, the range of \(L_A\) is
\[
\ran{L_A}=\{L_A(\vect{v})\in\R^m:\vect{v}\in\R^n\}
=\{A\vect{v}\in\R^m:\vect{v}\in\R^n\}
=\col{A}.
\]
From this, we can see that the notions of null space and range of a linear
transformation can be seen as generalizations to the concepts of null space and
column space for a matrix, respectively.

\item Like the null space and column space of a matrix, the null space and
range of a linear transformation are vector subspaces.

\begin{proposition}
\label{prp:null-ran-subspaces}
Let \(T:V\to W\) be a linear transformation. Then \(\nul{T}\) and \(\ran{T}\)
are vector subspaces of \(V\) and \(W\) respectively.
\end{proposition}
\begin{pf}
First consider \(\nul{T}\).
\underline{Closed under addition}: For any \(\vect{v},\vect{w}\in\nul{T}\), we have
\(T(\vect{v})=T(\vect{w})=\vect{0}\), thus
\(T(\vect{v}+\vect{w})=T(\vect{v})+T(\vect{w})=\vect{0}\). Hence,
\(\vect{v}+\vect{w}\in\nul{T}\).

\underline{Closed under scalar multiplication}: For any \(c\in\R\) and any
\(\vect{v}\in V\), we have \(T(c\vect{v})=cT(\vect{v})=\vect{0}\), thus
\(c\vect{v}\in\nul{T}\).

Next consider \(\ran{T}\).
\underline{Closed under addition}: For any \(\vect{v}',\vect{w}'\in\ran{T}\),
\(v'=T(\vect{v})\) and \(\vect{w}'=T(\vect{w})\) for some
\(\vect{v},\vect{w}\in V\). Thus,
\(\vect{v}'+\vect{w}'=T(\vect{v})+T(\vect{w})=T(\vect{v}+\vect{w})\in\ran{T}\).

\underline{Closed under scalar multiplication}: For any \(c\in\R\) and any
\(v'\in\ran{T}\), we have \(v'=T(\vect{v})\) for some \(\vect{v}\in V\). Hence,
\(c\vect{v}'=cT(\vect{v})=T(c\vect{v})\in\ran{T}\).
\end{pf}

\item To find the range of a linear transformation, the following result can be
of use.

\begin{proposition}
\label{prp:ran-spanning-set}
Let \(T:V\to W\) be a linear transformation and
\(\beta=\{\vect{v}_1,\dotsc,\vect{v}_n\}\) be a basis for \(V\). Then
\[
\ran{T}=\spn{\{T(\vect{v}_1),\dotsc,T(\vect{v}_n)\}}.
\]
\end{proposition}
\begin{pf}
``\(\supseteq\)'': For any
\(\vect{w}\in\spn{\{T(\vect{v}_1),\dotsc,T(\vect{v}_n)\}}\), we have for some
\(a_1,\dotsc,a_n\in\R\),
\[
\vect{w}=a_1T(\vect{v}_1)+\dotsb+a_nT(\vect{v}_n)
=T(a_1\vect{v}_1+\dotsb+a_n\vect{v}_n)\in\ran{T}.
\]

``\(\subseteq\)'': For any \(\vect{w}\in\ran{T}\), we know
\(\vect{w}=T(\vect{v})\) for some \(\vect{v}\in V\). By the spanning property
of basis, \(\vect{v}=b_1\vect{v}_1+\dotsb+b_n\vect{v}_n\) for some
\(b_1,\dotsc,b_m\in\R\). Thus
\[
\vect{w}=T(b_1\vect{v}_1+\dotsb+b_n\vect{v}_n)=b_1T(\vect{v}_1)+\dotsb+b_nT(\vect{v}_n)
\in\spn{\{T(\vect{v}_1),\dotsc,T(\vect{v}_n)\}}.
\]
\end{pf}

This result tells us that we can obtain the range of a linear transformation by
first finding a basis for the domain \(V\). Then, after applying the linear
transformation on each vector in the basis, the range would just be the span of
those transformed vectors.

\item Example: Let \(T:\R^3\to\R^2\) be defined by \(T\qty(\mqty[x\\ y\\
z])=\mqty[x-y\\ y-z]\). Then, the null space and the range of \(T\) are given
by:
\begin{itemize}
\item \emph{null space:} We first solve the system of linear equations
\[
\systeme{x-y=0, y-z=0}.
\]
The null space \(\nul{T}\) is just the solution set
\(\{(t,t,t)\in\R^3:t\in\R\}\).

\item \emph{range:} Consider the standard basis
\(\{\vect{e}_1,\vect{e}_2,\vect{e}_3\}\) for \(\R^3\). Then, by
\Cref{prp:ran-spanning-set},
\[
\ran{T}=\spn{\{T(\vect{e}_1),T(\vect{e}_2),T(\vect{e}_3)\}}
=\spn{\qty{\mqty[1\\ 0], \mqty[-1\\ 1], \mqty[0\\ 1]}}
=\R^2,
\]
where we get the last equality from the observation that \(\mqty[1\\ 0]\) and
\(\mqty[0\\ 1]\) already span \(\R^2\).
\end{itemize}

%\item The range of a linear transformation, as a vector subspace of the
%codomain, has a dimension bounded by the dimension of the codomain. But we can
%actually say even more.
%
%\begin{proposition}
%\label{prp:range-dim-bdd-dom-dim}
%Let \(T:V\to W\) be a linear transformation. Then \(\dim(\ran{T})\le\dim(V)\).
%\end{proposition}
%\begin{pf}
%Let \(\beta=\{\vect{v}_1,\dotsc,\vect{v}_n\}\) be a basis for \(V\). By
%\Cref{prp:ran-spanning-set},
%\(\ran{T}=\spn{\{T(\vect{v}_1),\dotsc,T(\vect{v}_n)\}}\). By the reduction
%approach in \Cref{subsect:construct-bases}, there is a linearly independent
%subset of \(\{T(\vect{v}_1),\dotsc,T(\vect{v}_n)\}\) that spans \(\ran{T}\),
%with cardinality less than or equal to \(\dim(V)\).  Thus
%\(\dim(\ran{T})\le\dim(V)\).
%\end{pf}

\item The concepts of null space and range of a linear transformation can be
neatly related by an important result called \emph{dimension formula}. Before
stating that, we first introduce some preliminary notions. Let \(T:V\to W\) be
a linear transformation.
\begin{itemize}
\item The \defn{nullity} of \(T\), denoted by \(\nulty{T}\), is the dimension
of \(\nul{T}\).
\item The \defn{rank} of \(T\), denoted by \(\rk{T}\), is the dimension of
\(\ran{T}\).
\end{itemize}
By \labelcref{it:null-ran-relate-matx-sp}, given an \(m\times n\) matrix \(A\)
and letting \(L_A:\R^n\to\R^m\) be defined by \(L_A(\vect{v})=A\vect{v}\) for
any \(\vect{v}\in\R^m\), we know that
\begin{itemize}
\item \(\nulty{L_A}=\dim(\nul{A})=\nulty{A}\).
\item \(\rk{L_A}=\dim(\col{A})=\rk{A}\).
\end{itemize}

\item The dimension formula is as follows.

\begin{theorem}[Dimension formula]
\label{thm:dim-fmla}
Let \(T:V\to W\) be a linear transformation. Then
\[
\nulty{T}+\rk{T}=\dim(V).
\]
\end{theorem}
\begin{pf}
Let \(k=\nulty{T}\) and \(m=\dim(V)\). Let \(\{\vect{v}_1,\dotsc,\vect{v}_k\}\)
be a basis for \(\nul{T}\), which is linearly independent in \(V\). By the
extension approach in \Cref{subsect:construct-bases}, we can find vectors
\(\vect{u}_1,\dotsc,\vect{u}_{m-k}\) such that
\(\{\vect{v}_1,\dotsc,\vect{v}_k,\vect{u}_1,\dotsc,\vect{u}_{m-k}\}\) is a
basis for \(V\).

By \Cref{prp:ran-spanning-set},
\begin{align*}
\ran{T}&=\spn{\{T(\vect{v}_1),\dotsc,T(\vect{v}_k),T(\vect{u}_1),\dotsc,T(\vect{u}_{m-k})\}} \\
&=\spn{\{\vect{0},\dotsc,\vect{0},T(\vect{u}_1),\dotsc,T(\vect{u}_{m-k})\}} \\
&=\spn{\{T(\vect{u}_1),\dotsc,T(\vect{u}_{m-k})\}}.
\end{align*}
Next, we shall show that \(\{T(\vect{u}_1),\dotsc,T(\vect{u}_{m-k})\}\) is
linearly independent as well, thus forming a basis for \(\ran{T}\). Consider
\[
a_1T(\vect{u}_1)+\dotsb+a_{m-k}T(\vect{u}_{m-k})=\vect{0},
\]
which implies \(T(a_1\vect{u}_1+\dotsb+a_{m-k}\vect{u}_{m-k})=\vect{0}\), thus
\(\vc{a_1\vect{u}_1+\dotsb+a_{m-k}\vect{u}_{m-k}}\in\nul{T}\). Then, by the spanning
property of the basis \(\{\vect{v}_1,\dotsc,\vect{v}_k\}\) for \(\nul{T}\), we
know
\[
\vc{a_1\vect{u}_1+\dotsb+a_{m-k}\vect{u}_{m-k}}=b_1\vect{v}_1+\dotsb+b_k\vect{v}_k
\]
for some \(b_1,\dotsc,b_k\in\R\). Rearranging it gives
\[
\vc{a_1\vect{u}_1+\dotsb+a_{m-k}\vect{u}_{m-k}}-b_1\vect{v}_1-\dotsb-b_k\vect{v}_k=\vect{0}.
\]

Since \(\{\vect{v}_1,\dotsc,\vect{v}_k,\vect{u}_1,\dotsc,\vect{u}_{m-k}\}\) is
a basis for \(V\) and hence linearly independent, we have
\[
a_1=\dotsb=a_{m-k}=-b_1=\dotsb=-b_k=0,
\]
which in particular implies \(a_1=\dotsb=a_{m-k}=0\). So
\(\{T(\vect{u}_1),\dotsc,T(\vect{u}_{m-k})\}\) is linearly independent, and
thus is a basis for \(\ran{T}\). This shows
\(\rk{T}=\dim(\ran{T})=m-k=\dim(V)-\nulty{T}\), as desired.
\end{pf}

\item \label{it:rank-nullity-thm} By setting \(T\) to be \(L_A:\R^n\to\R^m\)
defined by \(L_A(\vect{v})=A\vect{v}\) for any \(\vect{v}\in\R^n\) where \(A\)
is an \(m\times n\) matrix, the equality in the dimension formula reduces to
\[
\nulty{A}+\rk{A}=\dim(\R^n)=n=\text{number of columns of \(A\)}.
\]
This special case is known as the \emph{rank-nullity theorem}.
\end{enumerate}
\subsection{Injectivity and Surjectivity}
\begin{enumerate}
\item Next, we will consider the injectivity and surjectivity of a linear
transformation, which turn out to be related to the concepts of null space and
range.

\item A criterion for injectivity based on null space is as follows.

\begin{proposition}
\label{prp:inj-iff-null-sp-only-zero}
Let \(T:V\to W\) be a linear transformation. Then \(T\) is injective iff
\(\nul{T}=\{\vect{0}\}\).
\end{proposition}
\begin{pf}
``\(\Rightarrow\)'': Assume \(T\) is injective. Fix any \(\vect{v}\in\nul{T}\)
and we have \(T(\vect{v})=\vect{0}=T(\vect{0})\), where the last equality is a
property of linear transformation. By the injectivity of \(T\), it follows that
\(\vect{v}=\vect{0}\). This shows \(\nul{T}\subseteq \{\vect{0}\}\). But
another subset inclusion is immediate as \(T(\vect{0})=\vect{0}\) always, thus
\(\vect{0}\in\nul{T}\).

``\(\Leftarrow\)'': Assume \(\nul{T}=\{\vect{0}\}\). Fix any
\(\vect{v}_1,\vect{v}_2\in V\) with \(T(\vect{v}_1)=T(\vect{v}_2)\). Then,
\begin{align*}
T(\vect{v}_1)-T(\vect{v}_2)&=\vect{0} \\
\implies T(\vect{v}_1-\vect{v}_2)&=\vect{0} \\
\implies \vect{v}_1-\vect{v}_2&=\vect{0} &\text{(since \(\nul{T}=\{\vect{0}\}\))}\\
\implies \vect{v}_1&=\vect{v}_2.
\end{align*}
This shows \(T\) is injective.
\end{pf}

\begin{note}
Practically, it suffices to only prove that \(\nul{T}\subseteq \{\vect{0}\}\)
to show that \(T\) is injective, since the another subset inclusion is
immediate from the property that \(T(\vect{0})=\vect{0}\).
\end{note}

\item We can extend \Cref{prp:inj-iff-null-sp-only-zero} by utilizing the
dimension formula (\Cref{thm:dim-fmla}).

\begin{theorem}
\label{thm:lin-tran-inj-crit}
Let \(T:V\to W\) be a linear transformation. Then the following are equivalent.
\begin{enumerate}
\item \(T\) is injective.
\item \(\nul{T}=\{\vect{0}\}\).
\item \(\rk{T}=\dim(V)\).
\end{enumerate}
\end{theorem}
\begin{pf}
\(\text{(a)}\iff \text{(b)}\) follows from
\Cref{prp:inj-iff-null-sp-only-zero}. So it suffices to show that
\(\text{(b)}\iff\text{(c)}\).

\underline{\(\text{(b)}\implies \text{(c)}\)}: Assume \(\nul{T}=\{\vect{0}\}\).
Then, we have \(\nulty{T}=0\) and by \Cref{thm:dim-fmla},
\(\rk{T}=\dim(V)-0=\dim(V)\).

\underline{\(\text{(c)}\implies \text{(b)}\)}: Assume \(\rk{T}=\dim(V)\). By
\Cref{thm:dim-fmla}, \(\nulty{T}=\dim(V)-\rk{T}=\dim(V)-\dim(V)=0\). This means
\(\dim(\nul{T})=0\). But the only vector space with zero dimension is the zero
vector space \(\{\vect{0}\}\). Hence \(\nul{T}=\{\vect{0}\}\).
\end{pf}
\item To show that a linear transformation is \emph{not} injective, the
following result provides a convenient tool.

\begin{proposition}
\label{prp:suff-not-inj}
Let \(T:V\to W\) be a linear transformation. If \(\dim(V)>\dim(W)\), then \(T\)
is not injective.
\end{proposition}
\begin{pf}
We prove by contrapositive. Suppose \(T\) is injective. Then by
\Cref{thm:lin-tran-inj-crit}, \(\rk{T}=\dim(\ran{T})=\dim(V)\). But on the
other hand, as \(\ran{T}\) is a subspace of \(W\), we have
\(\rk{T}=\dim(\ran{T})\le\dim(W)\). This shows \(\dim(V)\le\dim(W)\).
\end{pf}

\item After discussing injectivity, we shall also discuss surjectivity in the
following result. It gives us a criterion for the bijectivity (i.e., both
injectivity and surjectivity) of a linear transformation.

\begin{theorem}
\label{thm:lin-tran-bij-crit}
Let \(T:V\to W\) be a linear transformation. Then \(T\) is bijective iff
\(\nul{T}=\{\vect{0}\}\) and \(\dim(V)=\dim(W)\).
\end{theorem}

\begin{pf}
``\(\Rightarrow\)'': Assume \(T\) is bijective. Then by the injectivity of
\(T\), we have \(\null{T}=\{\vect{0}\}\), thus \(\nulty{T}=0\). On the other
hand, by the surjectivity of \(T\), we have by definition \(\ran{T}=W\), so
\(\rk{T}=\dim(W)\). Finally, by \Cref{thm:dim-fmla}, we have
\(\dim(V)=\rk{T}+\nulty{T}=\rk{T}=\dim(W)\).


``\(\Leftarrow\)'' Assume \(\nul{T}=\{\vect{0}\}\) and \(\dim(V)=\dim(W)\). The
first condition implies that \(T\) is injective and \(\nulty{T}=0\). Next, by
\Cref{thm:dim-fmla}, \(\dim(W)=\dim(V)=\rk{T}+\nulty{T}=\rk{T}=\dim(\ran{T})\).
We claim that \(\ran{T}=W\).

\begin{pf}
First note that \(\ran{T}\subseteq W\). Assume to the contrary that \(\ran{T}\)
is a proper subset of \(W\). Then there exists \(\vect{w}\in
W\setminus\ran{T}\).  Let \(\beta\) be a basis for \(\ran{T}\) (which is
linearly independent in \(W\)). We know that \(|\beta|=\dim(\ran{T})=\dim(W)\)
and \(\spn{\beta}=\ran{T}\). Since \(\vect{w}\notin\ran{T}=\spn{\beta}\), by
\Cref{thm:add-vec-not-in-spn}, the union \(\beta\cup\{\vect{w}\}\) is linearly
independent in \(W\). But then this contradicts \Cref{prp:li-num-leq-dim} as
\(|\beta\cup\{\vect{w}\}|>\dim(W)\).
\end{pf}

Thus, \(T\) is surjective. Together with the injectivity of \(T\) shown
before, we conclude that \(T\) is bijective.
\end{pf}

\item As a corollary, we can obtain several more criteria for bijectivity.

\begin{corollary}
\label{cor:more-lt-bij-crit}
Let \(T:V\to W\) be a linear transformation. Then, the following are equivalent.
\begin{enumerate}
\item \(T\) is bijective.
\item \(T\) is injective and \(\dim(V)=\dim(W)\).
\item \(T\) is surjective and \(\dim(V)=\dim(W)\).
\end{enumerate}
\end{corollary}
\begin{pf}
\underline{\(\text{(a)}\iff\text{(b)}\)}:
\begin{align*}
&\hspace{2cm}\text{\(T\) is bijective} \\
&\iff\nul{T}=\{\vect{0}\}\text{ and }\dim(V)=\dim(W) &\text{(\Cref{thm:lin-tran-bij-crit})} \\
&\iff T\text{ is injective and }\dim(V)=\dim(W)&\text{(\Cref{prp:inj-iff-null-sp-only-zero})}.
\end{align*}

\underline{\(\text{(a)}\implies\text{(c)}\)}: Suppose that \(T\) is bijective. Then
\(T\) is immediately surjective and it suffices to show that
\(\dim(V)=\dim(W)\). But this just follows from \Cref{thm:lin-tran-bij-crit}.

\underline{\(\text{(c)}\implies \text{(a)}\)}: Suppose that \(T\) is surjective
and \(\dim(V)=\dim(W)\). It then suffices to show that \(T\) is injective. Due
to the surjectivity of \(T\), we have \(\rk{T}=\dim(\ran{T})=\dim(W)=\dim(V)\).
By dimension formula, we get
\[
\nulty{T}=\dim(V)-\rk{T}=0,
\]
which forces \(\nul{T}=\{\vect{0}\}\), hence \(T\) is injective.
\end{pf}

\item A bijective linear transformation is said to be an \defn{isomorphism}.
Given two vector spaces \(V\) and \(W\), if there exists an isomorphism from
\(V\) to \(W\), then we say that \(V\) is \defn{isomorphic} to \(W\).
(Intuitively, this means that \(V\) and \(W\) have the same ``structure'' in a
certain sense.)
\end{enumerate}
\subsection{Matrix Representations of Linear Transformations}
\label{subsect:lt-matx-rep}
\begin{enumerate}
\item From \labelcref{it:matx-mult-lt}, we know that the concepts of matrices
and linear transformations can be connected, by considering a linear
transformation \(L_A:\R^{n}\to\R^{m}\) associated to an \(m\times n\) matrix
\(A\). More generally, we can have a \emph{matrix representation} of any linear
transformation, which is the main topic to be discussed in
\Cref{subsect:lt-matx-rep}.

\item An \defn{ordered basis} is a basis equipped with a specific order. For an
ordered basis \(\beta=\{\vect{v}_1,\vect{v}_2,\dotsc,\vect{v}_n\}\) for a vector space
\(V\), we can say:
\begin{itemize}
\item \(\vect{v}_1\) is the \underline{first} one;
\item \(\vect{v}_2\) is the \underline{second} one;
\item ...
\item \(\vect{v}_n\) is the \underline{last} one.
\end{itemize}
Example: Consider two ordered bases \(\beta=\{\vect{e}_1,\vect{e}_2\}\) and
\(\beta'=\{\vect{e}_2,\vect{e}_1\}\) for \(\R^2\). Note that they are different
ordered bases.  Although we still use set notation, the order matters!

\item Before discussing matrix representation of a linear transformation, we
shall first introduce matrix representation of a \emph{vector}, as a
preliminary notion. Let \(\beta=\{\vect{v}_1,\dotsc,\vect{v}_n\}\) be an
ordered basis for a vector space \(V\). By \Cref{thm:basis-unique-lin-comb},
for any \(\vect{v}\in V\), we have
\[
\vect{v}=\vc{a_1}\vect{v}_1+\dotsb+\vc{a_n}\vect{v}_n
\]
for some unique scalars \(\vc{a_1,\dotsc,a_n}\in\R\). Using these unique
scalars, we define the \defn{coordinate vector of \(\vect{v}\) with respect to
\(\beta\)} by
\[
\defn{\([\vect{v}]_{\beta}\)}=\mqty[a_1\\ \vdots\\ a_n].
\]
Through this, we represent a vector using an \(n\times 1\) matrix.

In fact, the function \([\cdot]_{\beta}:V\to\R^n\) is a linear transformation.

\begin{pf}
Fix any \(\vect{u},\vect{v}\in V\). Suppose that we can write
\(\vect{u}=a_1'\vect{v}_1+\dotsb+a_n'\vect{v}_n\) and
\(\vect{v}=a_1\vect{v}_1+\dotsb+a_n\vect{v}_n\).

\underline{Addition}: Note that
\(\vect{u}+\vect{v}=(a_1'+a_1)\vect{v}_1+\dotsb+(a_n'+a_n)\vect{v}_n\). Thus,
\[
[\vect{u}+\vect{v}]_{\beta}
=\mqty[
a_1'+a_1\\
\vdots\\
a_n'+a_n
]
=\mqty[
a_1'\\ \vdots\\ a_n'
]
+\mqty[
a_1\\ \vdots\\ a_n
]
=[\vect{u}]_{\beta}+[\vect{v}]_{\beta}.
\]
\underline{Scalar multiplication}: For any \(c\in\R\),
\(c\vect{v}=(ca_1)\vect{v}_1+\dotsb+(ca_n)\vect{v}_n\), thus
\[
[c\vect{v}]_{\beta}=\mqty[ca_1\\ \vdots\\ ca_n]=c\mqty[a_1\\ \vdots\\ a_n]=c[\vect{v}]_{\beta}.
\]
\end{pf}

Furthermore, \([\cdot]_{\beta}\) is actually an
\emph{isomorphism}.

\begin{pf}
We shall use \Cref{thm:lin-tran-bij-crit}. It is immediate that
\(\dim(V)=\dim(\R^n)=n\), so it suffices to show that \(\nul{T}=\{\vect{0}\}\).
``\(\supseteq\)'' is immediate, so we only need to show \(\nul{T}\subseteq
\{\vect{0}\}\). Fix any \(\vect{v}\in\nul{T}\). Then we have
\([\vect{v}]_{\beta}=\vect{0}\), which implies
\(\vect{v}=0\vect{v}_1+\dotsb+0\vect{v}_n=\vect{0}\).
\end{pf}

Consequently, any vector space \(V\) with dimension \(n\) is isomorphic to
\(\R^n\).

\item Examples:
\begin{itemize}
\item Let \(\beta=\qty{\mqty[1\\ 1], \mqty[1\\ -1]}\) be an ordered basis for
\(\R^2\). Then, since \(\vect{v}=\mqty[2\\4] =3\mqty[1\\ 1]-1\mqty[1\\ -1]\), we have
\[
[\vect{v}]_{\beta}=\mqty[3\\ -1].
\]
\item Let \(\beta=\{\vect{e}_1,\vect{e}_2\}\) be an ordered basis for
\(\R^2\). Then, since \(\vect{v}=\mqty[2\\ 4]=2\vect{e}_1+4\vect{e}_2\), we have
\[
[\vect{v}]_{\beta}=\mqty[2\\ 4].
\]

\item Let \(\beta=\{\vect{e}_2,\vect{e}_1\}\) be an ordered basis for
\(\R^2\). Then, since \(\vect{v}=\mqty[2\\ 4]=4\vect{e}_2+2\vect{e}_1\), we have
\[
[\vect{v}]_{\beta}=\mqty[4\\ 2].
\]
\item Let \(\mathcal{P}_{3}\) denote the vector space of all polynomials of
degree at most 3, i.e.,
\(\mathcal{P}_3=\{a_0+a_1x+a_2x^2+a_3x^3:a_0,a_1,a_2,a_3\in\R\}\). Let
\(\beta=\{1,1+x,x^2,x^2+x^3\}\) be an ordered basis for \(\mathcal{P}_{3}\).
Then, since \(\vect{v}=1+x+x^2+x^3=0(1)+1(1+x)+0(x^2)+1(x^2+x^3)\), we have
\[
[\vect{v}]_{\beta}=\mqty[0\\ 1\\ 0\\ 1].
\]
\end{itemize}
\item Let us now consider matrix representation of a linear transformation. Let
\(T:V\to W\) be a linear transformation. Since there are two vector spaces
involved for a single linear transformation, the definition also involves two
ordered bases. We let \(\beta=\{\vect{v}_1,\dotsc,\vect{v}_n\}\) be an ordered
basis for \(V\) and \(\gamma=\{\vect{w}_1,\dotsc,\vect{w}_m\}\) be an ordered
basis for \(W\). For every \(j=1,\dotsc,n\), since \(T(\vect{v}_j)\in W\), by
\Cref{thm:basis-unique-lin-comb}, we have
\[
T(\vect{v}_j)=\vc{a_{1j}}\vect{w}_1+\dotsb+\vc{a_{mj}}\vect{w}_m
\]
for some unique scalars \(\vc{a_{1j},\dotsc,a_{mj}}\in\R\). From this we know that
\[
[T(\vect{v}_1)]_{\gamma}=\mqty[a_{11}\\ \vdots\\ a_{m1}],\quad
[T(\vect{v}_2)]_{\gamma}=\mqty[a_{12}\\ \vdots\\ a_{m2}],\qqtext{...},\quad
[T(\vect{v}_n)]_{\gamma}=\mqty[a_{1n}\\ \vdots\\ a_{mn}].
\]
Combining all these column vectors into a matrix gives the matrix
representation of \(T\). More precisely, the \defn{matrix representation of
\(T\) relative to \(\beta\) and \(\gamma\)} is the \(m\times n\) matrix
\[
\defn{\([T]_{\beta}^{\gamma}\)}
=\mqty[[T(\vect{v}_1)]_{\gamma}&\cdots&[T(\vect{v}_n)]_{\gamma}]
=[a_{ij}]=\mqty[a_{11}&\cdots&a_{m1}\\ \vdots&\ddots&\vdots\\
a_{m1}&\cdots&a_{mn}].
\]
\begin{note}
If \(\beta=\gamma\), we sometimes write \([T]_{\beta}\) instead of
\([T]_{\beta}^{\gamma}\).
\end{note}

\item Consider the special case where \(T\) is a linear transformation from
\(\R^n\) to \(\R^m\) and we use the standard bases in the matrix
representation. Let \(\beta\) and \(\beta'\) be the (ordered) standard bases
for \(\R^n\) and \(\R^m\) respectively. Then \([T]_{\beta}^{\beta'}\) is called
the \defn{standard matrix representation} of \(T\).

Example: Let \(T:\R^4\to\R^3\) be a linear transformation defined by
\[
T\qty(\mqty[x_1\\ x_2\\ x_3\\ x_4])=\mqty[x_1+x_3\\ x_2+x_3\\ x_1+x_4].
\]
Then the standard matrix representation of \(T\) is
\[
[T]_{\beta}^{\beta'}=\mqty[1&0&1&0\\ 0&1&1&0\\ 1&0&0&1],
\]
since:
\begin{itemize}
\item \(\displaystyle T\qty(\mqty[1\\ 0\\ 0\\ 0])=\mqty[1\\ 0\\
1]=1\vect{e}_1+0\vect{e}_2+1\vect{e}_3\),
\item \(\displaystyle T\qty(\mqty[0\\ 1\\ 0\\ 0])=\mqty[0\\ 1\\
0]=0\vect{e}_1+1\vect{e}_2+0\vect{e}_3\),
\item \(\displaystyle T\qty(\mqty[0\\ 0\\ 1\\ 0])=\mqty[1\\ 1\\
0]=1\vect{e}_1+1\vect{e}_2+0\vect{e}_3\),
\item \(\displaystyle T\qty(\mqty[0\\ 0\\ 0\\ 1])=\mqty[0\\ 0\\
1]=0\vect{e}_1+0\vect{e}_2+1\vect{e}_3\),
\end{itemize}
where \(\{\vect{e}_1,\vect{e}_2,\vect{e}_3\}\) is the standard basis for \(\R^3\).

\item Given a matrix representation of a linear transformation, we can recover
its explicit form through matrix multiplication.
\begin{theorem}
\label{thm:matx-rep-matx-mult}
Let \(T:V\to W\) be a linear transformation. Let \(\beta\) and \(\gamma\) be
ordered bases for \(V\) and \(W\) respectively. Then,
\[
[T(\vect{v})]_{\gamma}=[T]_{\beta}^{\gamma}[\vect{v}]_{\beta}
\]
for any \(\vect{v}\in V\).
\end{theorem}
\begin{pf}
Let \(\beta=\{\vect{v}_1,\dotsc,\vect{v}_m\}\) and
\(\gamma=\{\vect{w}_1,\dotsc,\vect{w}_n\}\).

We first establish the result for the vectors in \(\beta\). Firstly, note that
\([T(\vect{v}_i)]_{\gamma}\) is the \(i\)th column of \([T]_{\beta}^{\gamma}\)
and \([\vect{v}_i]_{\beta}=\vect{e}_i\) in \(\R^m\)\footnote{We can write
\(\vect{v}_i=0\vect{v}_1+\dotsb+0\vect{v}_{i-1}+1\vect{v}_i+0\vect{v}_{i+1}
+\dotsb+0\vect{v}_m\).}. Then, by definition of matrix multiplication,
\([T]_{\beta}^{\gamma}\vect{e}_i\) is the \(i\)th column of
\([T]_{\beta}^{\gamma}\). It follows that
\([T(\vect{v}_i)]_{\gamma}=[T]_{\beta}^{\gamma}[\vect{v}_i]_{\beta}\).

Next, we will establish the result for general vector in \(V\) to complete the
proof. For any \(\vect{v}\in V\), we have
\(\vect{v}=a_1\vect{v}_1+\dotsb+a_m\vect{v}_m\) for some unique scalars
\(a_1,\dotsc,a_m\). Then,
\begin{align*}
[T(\vect{v})]_{\gamma}&=[T(a_1\vect{v}_1+\dotsb+a_m\vect{v}_m)]_{\gamma} \\
&=[a_1T(\vect{v}_1)+\dotsb+a_mT(\vect{v}_m)]_{\gamma} \\
&=a_1[T(\vect{v}_1)]_{\gamma}+\dotsb+a_m[T(\vect{v}_m)]_{\gamma} \\
&=a_1[T]_{\beta}^{\gamma}\vect{e}_1+\dotsb+a_m[T]_{\beta}^{\gamma}\vect{e}_m \\
&=[T]_{\beta}^{\gamma}(a_1\vect{e}_1+\dotsb+a_m\vect{e}_m) \\
&=[T]_{\beta}^{\gamma}\mqty[a_1\\ \vdots\\ a_m] \\
&=[T]_{\beta}^{\gamma}[\vect{v}]_{\beta}.
\end{align*}
\end{pf}

\item Through matrix representations, we can establish some relationship
between operations between linear transformations and the matrix operations.
First we consider addition of linear transformations.

Let \(T:V\to W\) and \(T':V\to W\) be two linear transformations. Then \(T+T'\)
is defined to be a function from \(V\) to \(W\) given by
\((T+T')(\vect{v})=T(\vect{v})+T'(\vect{v})\). Note that
\(T+T'\) is also a linear transformation from \(V\) to \(W\).

\begin{pf}
\underline{Addition}: For any \(\vect{u},\vect{v}\in V\),
\begin{align*}
(T+T')(\vect{u}+\vect{v})&=T(\vect{u}+\vect{v})+T'(\vect{u}+\vect{v}) \\
&=T(\vect{u})+T'(\vect{u})+T(\vect{v})+T'(\vect{v}) \\
&=(T+T')(\vect{u})+(T+T')(\vect{v}).
\end{align*}
\underline{Scalar multiplication}: For any \(c\in\R\) and \(\vect{v}\in V\),
\[
(T+T')(c\vect{v})=T(c\vect{v})+T'(c\vect{v})
=c(T(\vect{v})+T'(\vect{v}))
=c(T+T')(\vect{v}).
\]
\end{pf}

\item As one may expect, addition of linear transformations corresponds to
matrix addition.
\begin{theorem}
\label{thm:lt-add-matx-add}
Let \(T,T':V\to W\) be two linear transformations. Let \(\beta\) and \(\gamma\)
be ordered bases for \(V\) and \(W\) respectively. Then,
\[
[T+T']_{\beta}^{\gamma}=[T]_{\beta}^{\gamma}+[T']_{\beta}^{\gamma}.
\]
\end{theorem}
\begin{pf}
For any \(\vect{v}\in V\),
\begin{align*}
[T+T']_{\beta}^{\gamma}[\vect{v}]_{\beta}&=[(T+T')(\vect{v})]_{\gamma}&\text{(\Cref{thm:matx-rep-matx-mult})} \\
&=[T(\vect{v})+T'(\vect{v})]_{\gamma} \\
&=[T(\vect{v})]_{\gamma}+[T'(\vect{v})]_{\gamma} \\
&=[T]_{\beta}^{\gamma}[\vect{v}]_{\beta}
+[T']_{\beta}^{\gamma}[\vect{v}]_{\beta}&\text{(\Cref{thm:matx-rep-matx-mult})} \\
&=([T]_{\beta}^{\gamma}+[T']_{\beta}^{\gamma})[\vect{v}]_{\beta}.
\end{align*}
As this holds for arbitrary \(\vect{v}\in V\), we conclude that
\[
[T+T']_{\beta}^{\gamma}=[T]_{\beta}^{\gamma}+[T']_{\beta}^{\gamma}.
\]
\end{pf}

\item Next, we consider composition of linear transformations.

Let \(T:V\to W\) and \(T':W\to X\) be two linear transformations. Then the
composition of \(T\) and \(T'\) is the function \(T'\circ T:V\to X\) given by
\[
(T'\circ T)(\vect{v})=T'(T(\vect{v}))
\]
for any \(\vect{v}\in V\). The composition \(T'\circ T\) is a linear
transformation from \(V\) to \(X\).

\begin{pf}
\underline{Addition}: For any \(\vect{u},\vect{v}\in V\),
\[
(T'\circ T)(\vect{u}+\vect{v})
=T'(T(\vect{u}+\vect{v}))
=T'(T(\vect{u})+T(\vect{v}))
=T'(T(\vect{u}))+T'(T(\vect{v}))
=(T'\circ T)(\vect{u})+(T'\circ T)(\vect{v}).
\]

\underline{Scalar multiplication}: For any \(\vect{v}\in V\) and \(c\in\R\),
\[
(T'\circ T)(c\vect{v})
=T'(T(c\vect{v}))
=T'(cT(\vect{v}))
=cT'(T(\vect{v}))
=c(T'\circ T)(\vect{v}).
\]
\end{pf}

\item It turns out that composition of linear transformations corresponds to
matrix multiplication. This explains the seemingly strange definition of matrix
multiplication.

\begin{theorem}
\label{thm:lt-compo-matx-mult}
Let \(T:V\to W\) and \(T':W\to X\) be two linear transformations. Let
\(\beta\), \(\gamma\), and \(\delta\) be ordered bases for \(V\), \(W\), and
\(X\) respectively. Then,
\[
[T'\circ T]_{\beta}^{\delta}=[T']_{\gamma}^{\delta}[T]_{\beta}^{\gamma}.
\]
\end{theorem}
\begin{pf}
For any \(\vect{v}\in V\),
\begin{align*}
[T'\circ T]_{\beta}^{\delta}[\vect{v}]_{\beta}
&=[(T'\circ T)(\vect{v})]_{\delta}&\text{(\Cref{thm:matx-rep-matx-mult})} \\
&=[T'(\vc{T(\vect{v})})]_{\delta} \\
&=[T']_{\gamma}^{\delta}[\vc{T(\vect{v})}]_{\gamma}&\text{(\Cref{thm:matx-rep-matx-mult})} \\
&=[T']_{\gamma}^{\delta}[T]_{\beta}^{\gamma}[\vect{v}]_{\beta}&\text{(\Cref{thm:matx-rep-matx-mult})}.
\end{align*}
As this holds for arbitrary \(\vect{v}\in V\), we conclude that
\[
[T'\circ T]_{\beta}^{\delta}=[T']_{\gamma}^{\delta}[T]_{\beta}^{\gamma}.
\]
\end{pf}

\item Apart from operations, there are also some correspondence in the
properties between linear transformations and their matrix representations. The
first one is invertibility.

\item Let \(T:V\to W\) be a linear transformation. \(T\) is said to be
\defn{invertible} if the inverse of \(T\) exists, i.e., there exists a function
\(T^{-1}:W\to V\) such that \(T^{-1}\circ T=\id_{V}\) and \(T\circ
T^{-1}=\id_{W}\). \(T^{-1}\) is called the \defn{inverse} of \(T\). From
MATH2012, we know that \(T\) is invertible iff \(T\) is bijective. So an
invertible linear transformation is just an isomorphism.

Note that the inverse of \(T\), namely \(T^{-1}\), is still a linear
transformation.

\begin{pf}
Fix any \(\vect{w}_1,\vect{w}_2\in W\). By the bijectivity of \(T\), there exist unique
\(\vect{v}_1,\vect{v}_2\in V\) such that \(T(\vect{v}_1)=\vect{w}_1\) and
\(T(\vect{v}_2)=\vect{w}_2\). Furthermore, we have
\(\vect{v}_1=T^{-1}(\vect{w}_1)\) and \(\vect{v}_2=T^{-1}(\vect{w}_2)\).

\underline{Addition}: Consider
\begin{align*}
T^{-1}(\vect{w}_1+\vect{w}_2)&=T^{-1}(T(\vect{v}_1)+T(\vect{v}_2)) \\
&=T^{-1}(T(\vect{v}_1))+T^{-1}(T(\vect{v}_2)) \\
&=\vect{v}_1+\vect{v}_2 \\
&=T^{-1}(\vect{w}_1)+T^{-1}(\vect{w}_2).
\end{align*}

\underline{Scalar multiplication}: For any \(c\in\R\), consider
\[
T^{-1}(c\vect{w}_1)
=T^{-1}(cT(\vect{v}_1)
=T^{-1}(T(c\vect{v}_1))
=c\vect{v}_1
=cT^{-1}(\vect{w}_1).
\]
\end{pf}
\item As one may expect, invertibility of a linear transformation corresponds
to the invertibility of its matrix representation.

\begin{theorem}
\label{thm:lt-inv-matx-inv}
Let \(T:V\to W\) be a linear transformation. Let \(\beta\) and \(\gamma\) be
ordered bases for \(V\) and \(W\) respectively. Then \(T\) is invertible iff
\([T]_{\beta}^{\gamma}\) is invertible.
\end{theorem}
\begin{pf}
Let \(m=\dim(V)\).

``\(\Rightarrow\)'': Assume that \(T\) is invertible. Then
\(\dim(W)=\dim(V)=m\). So both \(\beta\) and \(\gamma\) contain \(m\) vectors.
Thus, we have \([T^{-1}]_{\gamma}^{\beta}[T]_{\beta}^{\gamma}=[T^{-1}\circ
T]_{\beta}^{\beta}=[\id_{V}]_{\beta}^{\beta}=I_m\), and similarly,
\([T]_{\beta}^{\gamma}[T^{-1}]_{\gamma}^{\beta}=[\id_{W}]_{\gamma}^{\gamma}=I_m\).
This shows that \([T]_{\beta}^{\gamma}\) is invertible (and its inverse is given
by \([T^{-1}]_{\gamma}^{\beta}\)).

``\(\Leftarrow\)'' Assume that \([T]_{\beta}^{\gamma}\) is invertible. Let
\(A=[T]_{\beta}^{\gamma}\). Fix any \(\vect{w}\in W\). Write
\[
A^{-1}[\vect{w}]_{\gamma}=\mqty[a_1\\ \vdots\\ a_m].
\]
Also write the ordered bases as \(\beta=\{\vect{v}_1,\dotsc,\vect{v}_m\}\) and
\(\gamma=\{\vect{w}_1,\dotsc,\vect{w}_m\}\).  To construct the inverse of
\(T\), we define \(U:W\to V\) by
\(U(\vect{w})=a_1\vect{v}_1+\dotsb+a_m\vect{v}_m\). Note that by construction,
\[
[U(\vect{w})]_{\beta}=\mqty[a_1\\ \vdots\\ a_m]=A^{-1}[\vect{w}]_{\gamma}.
\]
We claim that \(U=T^{-1}\). To see this, note first that
\[
[U]_{\gamma}^{\beta}
=\mqty[[U(\vect{w}_1)]_{\beta}&\cdots&[U(\vect{w}_m)]_{\beta}]
=\mqty[A^{-1}[\vect{w}_1]_{\gamma}&\cdots&A^{-1}[\vect{w}_m]_{\gamma}]
=\mqty[A^{-1}\vect{e}_1&\cdots&A^{-1}\vect{e}_m]
=A^{-1}
\]
where \(\vect{e}_1,\dotsc,\vect{e}_m\) are standard vectors in \(\R^m\). By
\Cref{thm:lt-compo-matx-mult},
\[
[U\circ T]_{\beta}^{\beta}=\mqty[U]_{\gamma}^{\beta}\mqty[T]_{\beta}^{\gamma}
=A^{-1}A=I_m.
\]
Then by \Cref{thm:matx-rep-matx-mult}, we have for any \(\vect{v}\in V\),
\[
[(U\circ T)(\vect{v})]_{\beta}=[U\circ T]_{\beta}^{\beta}[\vect{v}]_{\beta}
=I_m[\vect{v}]_{\beta}
=[\vect{v}]_{\beta},
\]
implying that \((U\circ T)(\vect{v})=\vect{v}\), i.e., \(U\circ T=\id_{V}\).
Similarly, we can show that \(T\circ U=\id_{W}\). Hence, \(U=T^{-1}\).
\end{pf}

\item \label{it:matx-inv-by-inv-lt} As suggested in the proof of
\Cref{thm:lt-inv-matx-inv}, if \(T\) is invertible, then the inverse of the
matrix representation \([T]_{\beta}^{\gamma}\) is given by
\([T^{-1}]_{\gamma}^{\beta}\), namely the matrix representation of \(T^{-1}\).
This provides us another way to compute matrix inverse.

\item Next, we consider rank and nullity. It turns that a linear transformation
has the same rank and nullity as its matrix representation. \begin{note}
The terms ``rank'' and ``nullity'' have different meanings when applied to
linear transformations and matrices respectively.
\end{note}
\begin{theorem}
\label{thm:lt-matx-rep-same-rank-nul}
Let \(T\) be a linear transformation from \(V\) to \(W\). Let \(\beta\) and
\(\gamma\) be ordered bases for \(V\) and \(W\) respectively. Then,
\[
\rk{T}=\rk{[T]_{\beta}^{\gamma}}
\qqtext{and}
\nulty{T}=\nulty{[T]_{\beta}^{\gamma}}.
\]
\end{theorem}
\begin{pf}
(Partial) Note that it suffices to prove that
\(\rk{T}=\rk{[T]_{\beta}^{\gamma}}\), since it implies the latter by dimension
formula (\Cref{thm:dim-fmla}) and rank-nullity theorem
(\labelcref{it:rank-nullity-thm}).

Start with a basis \(\{\vect{v}_1,\dotsc,\vect{v}_r\}\) for \(\nul{T}\). By the
extension approach, we can add some vectors to form a basis
\(\{\vect{v}_1,\dotsc,\vect{v}_r,\vect{v}_{r+1},\dotsc,\vect{v}_n\}\) for
\(V\), denoted by \(\beta'\).

By \Cref{prp:ran-spanning-set},
\begin{align*}
\ran{T}&=\spn{\{T(\vect{v}_1),\dotsc,T(\vect{v}_r),T(\vect{v}_{r+1}),\dotsc,T(\vect{v}_n)\}} \\
&=\spn{\{\vect{0},\dotsc,\vect{0},T(\vect{v}_{r+1}),\dotsc,T(\vect{v}_n)\}} \\
&=\spn{\{T(\vect{v}_{r+1}),\dotsc,T(\vect{v}_n)\}}.
\end{align*}
By dimension formula (\Cref{thm:dim-fmla}), \(\dim(\ran{T})=\rk{T}=\dim
(V)-\nulty{T}=n-r\). Since\\ \(\{T(\vect{v}_{r+1}),\dotsc,T(\vect{v}_n)\}\) is a
spanning set for \(\ran{T}\) that contains \(n-r\) vectors, it is a basis for
\(\ran{T}\) by \Cref{thm:card-dim-spn-basis}.

Next, by the extension approach again, we can add some vectors to form a
basis\\
\(\{T(\vect{v}_{r+1}),\dotsc,T(\vect{v}_n),\vect{w}_1,\dotsc,\vect{w}_s\}\) for
\(W\) (with cardinality \(n-r+s\)), denoted by \(\gamma'\).

Note that the matrix representation
\begin{align*}
[T]_{\beta'}^{\gamma'}
&=\mqty[
[T(\vect{v}_1)]_{\gamma'}&\cdots&[T(\vect{v}_r)]_{\gamma'}
&[T(\vect{v}_{r+1})]_{\gamma'}&\cdots&[T(\vect{v}_{n})]_{\gamma'}
] \\
&=\mqty[
[\vect{0}]_{\gamma'}&\cdots&[\vect{0}]_{\gamma'}
&\vect{e}_1&\cdots&\vect{e}_{n-r}
]
&\text{(\(\vect{e}_1,\dotsc,\vect{e}_{n-r}\) are standard vectors in \(\R^{n-r+s}\))} \\
&=\mqty[
\vect{0}&\cdots&\vect{0}
&\vect{e}_1&\cdots&\vect{e}_{n-r}
] \\
&=\mqty[
O_{(n-r)\times r}&I_{n-r}\\
O_{s\times r}&O_{s\times (n-r)}
].
\end{align*}
Note that the rank of the matrix \([T]_{\beta'}^{\gamma'}\) is \(n-r\), since
\(\mqty[ O_{(n-r)\times r}&I_{n-r}\\ O_{s\times r}&O_{s\times (n-r)} ]\) is in
RREF and it has \(n-r\) leading ones. This is the same as the rank of \(T\),
given by \(\rk{T}=\dim(\ran{T})=n-r\) as well.

So far we have established the equality for a \emph{specific} choice of bases
for the matrix representation, namely \(\beta'\) and \(\gamma'\). To prove the
general case, we need to utilize the approach of \emph{change of coordinates},
to be discussed in \Cref{subsect:change-coord}. To be continued at
\labelcref{it:lt-rank-matx-pf-contd}.
\end{pf}
\end{enumerate}
\subsection{Change of Coordinates Matrix}
\label{subsect:change-coord}
\begin{enumerate}
\item Let \(\beta\) and \(\beta'\) be two ordered bases for a vector space
\(V\). Then the matrix \([\id_{V}]_{\beta}^{\beta'}\) is called a \defn{change
of coordinates matrix} from \(\beta\) to \(\beta'\).

Example: With \(\beta=\{\vect{e}_1,\vect{e}_2\}\) and \(\beta'=\qty{\mqty[1\\
2],\mqty[3\\ 4]}\) being two ordered bases for \(\R^2\), the change of
coordinates matrix from \(\beta\) to \(\beta'\) is
\[
[\id_{\R^2}]_{\beta}^{\beta'}=\mqty[1&3\\ 2&4].
\]

\item The following result tells us how \([\id_{V}]_{\beta}^{\beta'}\) is
related to ``change of coordinates''.

\begin{theorem}
\label{thm:change-coord-matx-prop}
Let \(\beta\) and \(\beta'\) be two ordered bases for a vector space \(V\). Let
\(Q=[\id_{V}]_{\beta}^{\beta'}\). Then,
\begin{enumerate}
\item \(Q\) is invertible.
\item (change of coordinates) For any \(\vect{v}\in V\),
\([\vect{v}]_{\beta'}=Q[\vect{v}]_{\beta}\).
\end{enumerate}
\end{theorem}
\begin{note}
From the \emph{change of coordinates} property, we see that multiplying the
change of coordinates matrix \(Q\) does change the coordinate vector from
\([\vect{v}]_{\beta}\) (with respect to \(\beta\)) to \([\vect{v}]_{\beta'}\)
(with respect to \(\beta'\)).
\end{note}

\begin{pf}
\begin{enumerate}
\item Note that \(\id_{V}\) is invertible, so by \Cref{thm:lt-inv-matx-inv},
the matrix representation \(Q=[\id_{V}]_{\beta}^{\beta'}\) is invertible as well.
\item For any \(\vect{v}\in V\), we have
\(Q[\vect{v}]_{\beta}=[\id_{V}]_{\beta}^{\beta'}[\vect{v}]_{\beta}
\overset{\text{thm.\ \labelcref{thm:matx-rep-matx-mult}}}{=}[\id_{V}(\vect{v})]_{\beta'}
=[\vect{v}]_{\beta'}\).
\end{enumerate}
\end{pf}

\item Using change of coordinates matrix, we also change the matrix
representation of a linear transformation.

\begin{theorem}
\label{thm:change-lt-matx-rep}
Let \(\beta\) and \(\beta'\) be two ordered bases for a vector space \(V\). Let
\(T:V\to V\) be a linear transformation and let
\(Q=[\id_{V}]_{\beta}^{\beta'}\). Then,
\[
[T]_{\beta}^{\beta}=Q^{-1}[T]_{\beta'}^{\beta'}Q.
\]
\end{theorem}
\begin{warning}
We do \underline{not} have \([T]_{\beta'}^{\beta'}=Q^{-1}[T]_{\beta}^{\beta}Q\)
with this definition of \(Q\).
\end{warning}

\begin{pf}
Note first that
\(Q^{-1}=\qty([\id_{V}]_{\beta}^{\beta'})^{-1}=[\id_{V}]_{\beta'}^{\beta}\) by
\labelcref{it:matx-inv-by-inv-lt}. Then, consider:
\begin{align*}
Q^{-1}[T]_{\beta'}^{\beta'}Q
&=[\id_{V}]_{\beta'}^{\beta}[T]_{\beta'}^{\beta'}[\id_{V}]_{\beta}^{\beta'} \\
&=[\id_{V}]_{\beta'}^{\beta}[T\circ\id_{V}]_{\beta}^{\beta'}&\text{(\Cref{thm:lt-compo-matx-mult})} \\
&=[\id_{V}]_{\beta'}^{\beta}[T]_{\beta}^{\beta'} \\
&=[\id_{V}\circ T]_{\beta}^{\beta}&\text{(\Cref{thm:lt-compo-matx-mult})} \\
&=[T]_{\beta}^{\beta}.
\end{align*}
\end{pf}

\item \Cref{thm:change-lt-matx-rep} is somehow related to the following
definition about \emph{similar} matrices. An \(n\times n\) matrix \(A\) is said
to be \defn{similar} to an \(n\times n\) matrix \(B\) if there exists an
invertible matrix \(Q\) such that \(B=Q^{-1}AQ\). In view of
\Cref{thm:change-lt-matx-rep,thm:change-coord-matx-prop}, we know that the two
matrix representations \([T]_{\beta'}^{\beta'}\) and \([T]_{\beta}^{\beta}\)
are similar.

\item \label{it:matx-similar-equiv-relate}
As one may expect, matrix similarity is an equivalence relation.

\begin{pf}
Write \(A\sim B\) if \(A\) is similar to \(B\).

\underline{Reflexive}: For any \(n\times n\) matrix \(A\), we can write
\(A=I_n^{-1}AI_n\), so \(A\sim A\).

\underline{Symmetric}: For any \(n\times n\) matrices \(A\) and \(B\), we have
\[
A\sim B
\implies B=Q^{-1}AQ
\implies A=QBQ^{-1}=(\vc{Q^{-1}})^{-1}B\vc{Q^{-1}}
\implies B\sim A
\]
where \(Q\) is an invertible matrix.

\underline{Transitive}: For any \(n\times n\) matrices \(A\), \(B\), and \(C\),
assume \(A\sim B\) and \(B\sim C\). Then, there exist invertible matrices \(P\)
and \(Q\) such that \(B=P^{-1}AP\) and \(C=Q^{-1}BQ\). Then,
\[
C=Q^{-1}(P^{-1}AP)Q=(PQ)^{-1}A(PQ),
\]
so \(A\sim C\).
\end{pf}

\item As suggested by the name, two similar matrices \(A\) and \(B\) indeed
share a number of common properties:
\begin{enumerate}
\item \label{it:sim-matx-same-rk} \emph{same rank:} Note that
\begin{align*}
\rk{B}&=\rk{Q^{-1}AQ} \\
&=\rk{AQ}&\text{(\Cref{thm:mult-inv-matx-preserv-dim})} \\
&=\rk{Q^{T}A^{T}} \\
&=\rk{A^{T}}&\text{(\Cref{thm:mult-inv-matx-preserv-dim})} \\
&=\rk{A}&\text{(\Cref{cor:matx-trans-same-rk})}.
\end{align*}
\item \label{it:sim-matx-same-det} \emph{same determinant:} Note that \(\det
B=\det(Q^{-1}AQ)=\det(Q^{-1})\det A\det Q=(\det Q)^{-1}\det Q\det A=\det A\).
\end{enumerate}

\item \label{it:lt-rank-matx-pf-contd} Using a similar idea as the proof of
\Cref{thm:change-lt-matx-rep}, we can complete the rest of the proof for
\Cref{thm:lt-matx-rep-same-rank-nul}:

\begin{pf}
(Continued) Given any ordered bases \(\beta\) and \(\gamma\) for \(V\) and
\(W\) respectively, we have by \Cref{thm:lt-compo-matx-mult},
\[
[\id_{W}]_{\gamma}^{\gamma'}[T]_{\beta}^{\gamma}[\id_{V}]_{\beta'}^{\beta}
=[T]_{\beta'}^{\gamma'}.
\]
Since both \([\id_{W}]_{\gamma}^{\gamma'}\) and \([\id_{V}]_{\beta'}^{\beta}\)
are invertible matrices (as identity transformation is invertible),
\([T]_{\beta}^{\gamma}\) and \([T]_{\beta'}^{\gamma'}\) are similar. Thus, by
\labelcref{it:sim-matx-same-rk}, we have
\[
\rk{[T]_{\beta}^{\gamma}}=\rk{[T]_{\beta'}^{\gamma'}}.
\]
So the general case follows from the proven specific case.
\end{pf}

\end{enumerate}
