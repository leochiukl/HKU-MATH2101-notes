\subsection*{\Cref{sect:linear-transformations}}
\item \Cref{prp:lt-zero-to-zero-preserv-lincomb}: preservation of zero and linear combinations after linear transformation
\item \Cref{thm:construct-lt-from-basis}: construction of linear transformation from basis
\item \labelcref{it:null-ran-relate-matx-sp}: null space and range of a linear transformation associated to a matrix
\item \Cref{prp:null-ran-subspaces}: null space and range of a linear transformation are vector subspaces
\item \Cref{prp:ran-spanning-set}: range is the spanning set of linearly transformed vectors in basis
\item \Cref{thm:dim-fmla}: dimension formula
\item \labelcref{it:rank-nullity-thm}: rank-nullity theorem
\item \Cref{prp:inj-iff-null-sp-only-zero}: relationship between injectivity and null space
\item \Cref{thm:lin-tran-inj-crit}: criteria for injectivity of a linear transformation
\item \Cref{prp:suff-not-inj}: sufficient condition for non-injectivity of a linear transformation
\item \Cref{thm:lin-tran-bij-crit}: criterion for bijectivity of a linear transformation about null space and dimensions
\item \Cref{cor:more-lt-bij-crit}: criteria for bijectivity of a linear transformation
\item \Cref{thm:matx-rep-matx-mult}: relationship between matrix representation and matrix-vector product
\item \Cref{thm:lt-add-matx-add}: addition of linear transformations
corresponds to addition of matrix representations
\item \Cref{thm:lt-compo-matx-mult}: composition of linear transformations
corresponds to multiplication of matrix representations
\item \Cref{thm:lt-inv-matx-inv}: invertibility of a linear transformation
corresponds to invertibility of its matrix representation
\item \labelcref{it:matx-inv-by-inv-lt}: finding matrix inverse from the inverse of linear transformation
\item \Cref{thm:lt-matx-rep-same-rank-nul}: equality of ranks/nullities of
linear transformation and its matrix representation
\item \Cref{thm:change-coord-matx-prop}: properties of change of coordinates matrix
\item \Cref{thm:change-lt-matx-rep}: method for changing matrix representation of linear transformation
\item \labelcref{it:matx-similar-equiv-relate}: matrix similarity is an equivalence relation
\item \labelcref{it:sim-matx-same-rk}: similar matrices have the same rank
\item \labelcref{it:sim-matx-same-det}: similar matrices have the same determinant


