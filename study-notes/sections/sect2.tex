\section{Systems of Linear Equations}
\label{sect:sle}
\subsection{Basic Notions and Methods for Systems of Linear Equations}
\begin{enumerate}
\item\label{it:sle} A \defn{system of \(m\) linear equations in \(n\) unknowns} (or
\defn{variables}) \(x_1,,\dotsc,x_n\) is a family of equations of the form
\[
\begin{cases}
a_{11}x_1+\dotsb+a_{1n}x_{n}=b_1\\
a_{21}x_1+\dotsb+a_{2n}x_{n}=b_2\\
\hspace{2cm}\vdots\\
a_{m1}x_1+\dotsb+a_{mn}x_{n}=b_m\\
\end{cases}
\]
where \(a_{ij}\)'s and \(b_k\)'s are some real constants.

\item For the system in \labelcref{it:sle}, the \(m\times n\) matrix
\([a_{ij}]\) is called the \defn{coefficient matrix} of the system, while the
matrix
\[
\left[
\begin{array}{@{}cccc|c@{}}
a_{11}&a_{12}&\cdots&a_{1n}&b_1\\
a_{21}&a_{22}&\cdots&a_{2n}&b_2\\
\vdots&\vdots&\ddots&\vdots&\vdots\\
a_{m1}&a_{m2}&\cdots&a_{2n}&b_m
\end{array}
\right]
\]
is called the \defn{augmented matrix} of the system.

\begin{note}
The vertical line only serves for decorational purpose (entries on its right
are the values appearing on the right of ``\(=\)'' in the system), and it does
not affect the nature of the matrix, i.e., the vertical line can be ignored
when interpreting the matrix.
\end{note}

\item Writing \(A=[a_{ij}]\), \(\vect{x}=\mqty[x_1\\\vdots\\ x_n]\), and
\(\vect{b}=\mqty[b_1\\ \vdots\\ b_m]\), the system in \labelcref{it:sle} can be
rewritten simply as
\[
A\vect{x}=\vect{b}.
\]
Then, we can observe that solving the system is just the same as solving this
matrix equation. Hence, a \emph{solution} to the system is just a column vector
\(\vect{v}\in\R^m\) such that \(A\vect{v}=\vect{b}\). The \defn{solution set}
of the system is the set \(\{\vect{v}\in\R^m:A\vect{v}=\vect{b}\}\), i.e., the
set of all solutions to the system.

\item \label{it:matrix-inv-solve-sle}
Next, we will introduce two approaches for solving a system of linear
equations. The first one is related to the matrix inverse, and is applicable
when the coefficient matrix \(A\) is an \emph{invertible square matrix}.

To solve the system \(A\vect{v}=\vect{b}\), multiplying both sides by
\(A^{-1}\) gives
\[
\vect{x}=A^{-1}\rc{A\vect{x}}=\boxed{A^{-1}\rc{\vect{b}}}.
\]
This suggests a way of solving the system. Furthermore, we have the following
result which guarantees the \emph{uniqueness} of the solution in this case.

\begin{proposition}
\label{prp:inv-sle-unique}
Let \(A\) be an \(n\times n\) \emph{invertible} matrix and let \(\vect{b}\in\R^n\) be a column
vector. Then, the system of linear equations \(A\vect{x}=\vect{b}\) must have a
unique solution.
\end{proposition}
\begin{pf}
This follows from the uniqueness of matrix inverse. Particularly, the unique
solution is given by \(\vect{x}=A^{-1}\vect{b}\).
\end{pf}

\item Another approach for solving a system of linear equations is also only
applicable when \(A\) is an invertible square matrix. However, it often
requires fewer computations than the matrix inverse approach, and so is more
frequently used. It is known as \emph{Cramer's rule}:
\begin{theorem}[Cramer's rule]
\label{thm:cramer-rule}
Let \(A\) be an \(n\times n\) invertible matrix and let \(\vect{b}\in\R^n\) be
a column vector. Then, the (unique) solution of the system
\[
A\mqty[x_1\\ \vdots\\ x_n]=\vect{b}
\]
is given by
\[
x_i=\frac{\det A_i}{\det A}
\]
where \(A_i\) is the matrix obtained by replacing the \(i\)th column of \(A\)
by \(\vect{b}\), for any \(i=1,\dotsc,n\).
\end{theorem}
\begin{pf}
By \cref{cor:matrix-inv-fmla}, we have
\[
A^{-1}=\frac{1}{\det A}\adj A.
\]
We also know that the solution of the system is \(\vect{x}=A^{-1}\vect{b}\), so
it suffices to show that for any \(i=1,\dotsc,n\),
\[
\frac{\det A_i}{\det A}=\qty(\frac{1}{\det A}(\adj A)\vect{b})_{i1},
\]
or
\[
\det A_i=\qty[(\adj A)\vect{b}]_{i1},
\]
where RHS denotes the \((i,1)\)-entry of the \(n\times 1\) matrix
\((\adj A)\vect{b}\).

First fix any \(i=1,\dotsc,n\). By the definition of matrix multiplication,
\[
\qty[(\adj A)\vect{b}]_{i1}
=C_{1i}b_1+\dotsb+C_{ni}b_n
\]
where \(C_{ji}=(-1)^{j+i}\widetilde{A}_{ji}\) is the \((j,i)\)-cofactor of
\(A\). On the other hand, performing cofactor expansion for \(\det A_i\) along
the \(i\)th column gives also
\[
\det A_i=b_1C_{1i}+\dotsb+b_nC_{ni}.
\]
This shows the desired equality.
\end{pf}
\end{enumerate}
\subsection{Elementary Row Operations}
\begin{enumerate}
\item \label{it:motivate-ero}
To motivate the concept of \emph{elementary row operations}. Consider the
following system:
\[
\systeme{
x_1+x_2=3,
x_2=2
}
\]
It is straightforward to solve this system since we can simply put \(x_2=2\)
into the first equation to get \(x_1=1\). However, a system is generally not
that easy to solve, and this leads us to consider some systematic methods to
``transform'' an arbitrary system into a ``simple'' system like the above one.
It turns out that it is possible by performing the so-called \emph{elementary
row operations}.

\item There are three types of \defn{elementary row operations} (EROs) to be
performed on a matrix:
\begin{itemize}
\item \defn{type I ERO}: interchanging two different rows
\item \defn{type II ERO}: multiply a row by a \emph{nonzero} scalar
\item \defn{type III ERO}: adding a scalar multiple of a row to another row
\end{itemize}
Here, we shall use \(\vect{r}_1\), \(\vect{r}_{2}\), \ldots, \(\vect{r}_{m}\)
to denote the \(m\) rows of an \(m\times n\), from top to bottom. We have the
following notations for the three types of EROs:
\begin{itemize}
\item (type I) \(\vect{r}_{i}\leftrightarrow \vect{r}_{j}\): interchanging the
\(i\)th and the \(j\)th row (``putting \(\vect{r}_{i}\) to the \(j\)th row
\(\vect{r}_{j}\), and putting \(\vect{r}_{j}\) to the \(i\)th row
\(\vect{r}_{i}\)'')
\item (type II) \(k\vect{r}_{i}\to \vect{r}_{i}\): multiplying the \(i\)th row
by a \emph{nonzero} scalar \(k\) (``putting \(k\vect{r}_{i}\) back to the
\(i\)th row \(\vect{r}_{i}\)'')
\item (type III) \(k\vect{r}_{j}+\vect{r}_{i}\to \vect{r}_{i}\): adding \(k\)
times the \(j\)th row to the \(i\)th row (``putting
\(k\vect{r}_{j}+\vect{r}_{i}\) to the \(i\)th row \(\vect{r}_{i}\)'')
\end{itemize}
\end{enumerate}
\subsection{Elementary Matrices}
\begin{enumerate}
\item To utilize the results for \emph{matrices} in \cref{sect:matrix-algebra}
for EROs, we need to find a way to associate each ERO with a \emph{matrix}. It
turns out that \emph{elementary matrices} are able to capture the essence of
EROs.

\item An \defn{elementary matrix} is a square matrix obtained by performing
\emph{exactly one} ERO on an identity matrix. An elementary matrix is of type
I, II, or III if the ERO performed on the identity matrix is of type I, II, or
III respectively.

Examples:
\begin{itemize}
\item \(\mqty[1&0\\ 0&1]\) is a type I elementary matrix.
\item \(\mqty[1&0&0\\0&3&0\\ 0&0&1]\) is a type II elementary matrix.
\item \(\mqty[1&2\\ 0&1]\) is a type III elementary matrix.
\end{itemize}

\item The following theorem suggests the connection between elementary matrix
and ERO:
\begin{theorem}
\label{thm:elementary-matrix-ero-relate}
Let \(A\) be an \(m\times n\) matrix. If \(B\) is obtained from \(A\) by
performing an ERO, then there exists an \(m\times m\) elementary matrix \(E\)
such that \(B=EA\), and the matrix \(E\) can be obtained by performing the same
ERO on \(I_m\).

Conversely, if \(E\) is an \(m\times m\) elementary matrix, then \(EA\) is the
matrix obtained from \(A\) by performing the ERO corresponding to \(E\).
\end{theorem}
\item An important property of elementary matrix is \emph{invertibility}:
\begin{proposition}
\label{prp:elementary-matrix-inv}
Elementary matrices are invertible. Furthermore, the inverse of an elementary
matrix is an elementary matrix of the same type.
\end{proposition}
\begin{pf}
It follows from \cref{thm:elementary-matrix-ero-relate} and the property that
the effect from any ERO can be cancelled out by performing an ERO of the same
type. More explicitly, the reverse processes of all kinds of ERO are given as
follows.
\begin{itemize}
\item The reverse process of \(\vect{r}_{i}\leftrightarrow \vect{r}_{j}\) is
\(\vect{r}_{i}\leftrightarrow \vect{r}_{j}\).
\item The reverse process of \(c\vect{r}_{i}\to \vect{r}_{i}\) is
\(\displaystyle \frac{1}{c}\vect{r}_{i}\to\vect{r}_{i}\). (Note that \(c\ne 0\).)
\item The reverse process of \(c\vect{r}_{j}+\vect{r}_{i}\to \vect{r}_{i}\) is
\(-c\vect{r}_{j}+\vect{r}_{i}\to \vect{r}_{i}\).
\end{itemize}
\end{pf}
\end{enumerate}

\subsection{(Reduced) Row Echelon Form}
\begin{enumerate}
\item Recall the simple-to-solve system in \labelcref{it:motivate-ero}:
\[
\systeme{
x_1+x_2=3,
x_2=2
}
\]
In augmented matrix form, it can be expressed as
\[
\left[\begin{array}{@{}cc|c@{}}
1&1&3\\
0&1&2
\end{array}\right].
\]
Consider yet another simple-to-solve system of three equations:
\[
\systeme{
x_1+2x_3=3,
x_2+x_3=2,
x_3=1
}.
\]
The corresponding augmented matrix is
\[
\left[\begin{array}{@{}ccc|c@{}}
1&0&2&3\\
0&1&1&2\\
0&0&1&1\\
\end{array}\right].
\]
A common feature for these augmented matrices of these simple-to-solve systems
is that there appears to be a ``triangle of zeros'' located at the lower left
corner. We can observe that if the augmented matrix of a system has this kind
of nice ``format'', then the system is quite easy to solve.

\item To formalize this notion, we introduce the concept of \emph{row echelon
form}. A matrix is said to be in \defn{row echelon form} (REF) if the
following conditions are all satisfied:
\begin{itemize}
\item All rows having only zero entries (called \defn{zero rows}), if any, lie
at the bottom of the matrix (i.e., below every nonzero row).
\item The \defn{leading entry} (i.e., the leftmost nonzero entry) in a
nonzero row is on the right of the leading entry of every row above.
\end{itemize}
Examples and non-examples:
\begin{itemize}
\item \(
\left[\begin{array}{@{}cc|c@{}}
\rc{1}&1&3\\
0&\rc{1}&2
\end{array}\right]
\) is in REF.
\item \(
\left[\begin{array}{@{}ccc|c@{}}
\rc{1}&0&2&3\\
0&\rc{1}&1&2\\
0&0&\rc{1}&1\\
\end{array}\right]
\) is in REF.

\item \(\left[\begin{array}{@{}cc|c@{}}
\rc{1}&1&3\\
0&\rc{1}&2\\
0&0&0\\
\end{array}\right]
\) is in REF.

\item \(\left[\begin{array}{@{}cc|c@{}}
1&1&3\\
0&0&0\\
0&1&2\\
\end{array}\right]
\) is \emph{not} in REF. This is because the zero row (second row) is not at
the bottom of the matrix.

\item \(\mqty[\rc{1}&2&3&4\\ 0&0&\rc{2}&1\\ 0&0&0&\rc{1}]\) is in REF.

\item \(I_n\) and any zero matrix are in REF.

\item \(\mqty[1&2&3&4\\ 0&0&\blc{2}&1\\ 0&0&\rc{1}&1]\) is \emph{not} in REF.
This is because the leading entry \(\rc{1}\) in the third row is not on the
right of the leading entry \(\blc{2}\) in the second row.
\end{itemize}

\item Consider the following system
\[
\systeme{
x_1=3,
x_2=2,
x_3=1
}.
\]
Its augmented matrix is
\[
\left[\begin{array}{@{}ccc|c@{}}
1&0&0&3\\
0&1&0&2\\
0&0&1&1\\
\end{array}\right].
\]

It is very straightforward to obtain the solution to this system. Indeed, in
this case we can even directly ``read out'' the solution. Hence, we would also
like to introduce a notion to describe this kind of \emph{very}
easy-to-solve system. It is known as \emph{reduced row echelon form}.

\item A matrix is said to be in \defn{reduced row echelon form} (RREF) if it is
in REF with the following additional conditions satisfied:
\begin{itemize}
\item The leading entry of every nonzero row is \(1\) (called \defn{leading one}).
\item For each leading one, all other entries in the same column are \(0\).
\end{itemize}
Examples and non-examples:
\begin{itemize}
\item \(
\left[\begin{array}{@{}ccc|c@{}}
\rc{1}&0&0&3\\
0&\rc{1}&0&2\\
0&0&\rc{1}&1\\
\end{array}\right]
\) is in RREF.

\item \(
\left[\begin{array}{@{}cc|c@{}}
\rc{1}&1&3\\
0&\rc{1}&2
\end{array}\right]
\) is \emph{not} in RREF.

\item \(\mqty[\rc{1}&2&0&4\\ 0&0&\rc{1}&1\\ 0&0&0&0]\) is in RREF.
\item \(\mqty[\rc{1}&0&3\\ 0&0&\rc{1}]\) is \emph{not} in RREF.

\item \(I_n\) and any zero matrix are in RREF.
\end{itemize}

\item Apart from the practical application of solving systems of linear
equations, the concept of RREF is also useful \emph{theoretically}. It turns
that for a square matrix in RREF, there are some extra criteria for
matrix invertibility:
\begin{theorem}
\label{thm:rref-matx-inv-equiv}
Let \(A\) be an \(n\times n\) matrix in RREF. Then the following statements are
equivalent.
\begin{enumerate}
\item \(A\) is invertible.
\item \(A\) has \(n\) leading ones.
\item \(A=I_n\).
\end{enumerate}
\end{theorem}
\begin{pf}
To prove statements of this form, a method is to prove a ``cycle of
implications'' (logically speaking). Here we will prove that
\(\text{(a)}\implies \text{(b)}\implies \text{(c)}\implies \text{(a)}\).

\underline{\(\text{(a)}\implies \text{(b)}\)}: Assume that \(A\) is invertible.
Then, there must be no zero row in \(A\); otherwise, performing cofactor
expansion along the zero row suggests that \(\det A=0\), which means \(A\) is
not invertible, contradiction. Hence, each row must have a leading one, and so
\(A\) has \(n\) leading ones in total.

\underline{\(\text{(b)}\implies \text{(c)}\)}: Assume that \(A\) has \(n\)
leading ones. Since \(A\) has \(n\) rows, it implies that each row of \(A\) has
a leading one. Now, as \(A\) is in RREF, the leading ones have to occupy all
the diagonal entries, thus all non-diagonal entries are zero. So,
\(A\) must be \(I_n\).

\underline{\(\text{(c)}\implies \text{(a)}\)}: It follows from the fact that
\(I_n\) is invertible.
\end{pf}
\end{enumerate}
\subsection{Gaussian Elimination}
\begin{enumerate}
\item As noted previously, one can solve a system whose augmented matrix is in
RREF very easily. It turns out that there is a systematic method for
``converting''/``reducing'' \emph{any} system of linear equations into a system
whose augmented matrix is in RREF. This is known as \emph{Gaussian
elimination}, which is an important tool for solving a system of linear
equation.

\item But before discussing Gaussian elimination, we first introduce a
preliminary notion: \emph{row equivalence}. For two \(m\times n\) matrices
\(A\) and \(B\), the matrix \(A\) is said to be \defn{row equivalent} to \(B\)
if there exist elementary matrices \(E_1,\dotsc,E_k\) such that \(E_k\dotsb
E_1A=B\) (or equivalently, \(B\) can be obtained from \(A\) by performing some
EROs).

\item \label{it:row-equiv-equiv-relation}
Row equivalence is indeed an \emph{equivalence relation} in mathematical sense.
To see this, consider the following. (\(A\), \(B\), and \(C\) denote arbitrary
matrices below.)
\begin{itemize}
\item reflexive: \(A\) is row equivalent to \(A\). For example, we can
interchange two rows of \(A\) twice to get back \(A\).

\item symmetric: Suppose that \(A\) is row equivalent to \(B\), i.e., \(E_k\dotsb
E_1A=B\) for some elementary matrices \(E_1,\dotsc,E_k\). Since the
inverse of any elementary matrix exists and is also elementary, by writing
\[
A=E_1^{-1}\dotsb E_k^{-1}B,
\]
we see that \(B\) is row equivalent to \(A\).

\item transitive: Assume that \(A\) is row equivalent to \(B\), and \(B\) is row
equivalent to \(C\). Then, we have
\[
E_{k}\dotsb E_{1}A=B\qqtext{and}
F_{\ell}\dotsb F_1B=C
\]
for some elementary matrices \(E_1,\dotsc,E_k,F_1,\dotsc,F_{\ell}\). But then
we can write
\[
F_{\ell}\dotsb F_1E_{k}\dotsb E_{1}A=C,
\]
so \(A\) is row equivalent to \(C\).
\end{itemize}
\item An important result that serves as the foundation for the Gaussian
elimination is the following.

\begin{theorem}
\label{thm:row-equiv-same-sol}
Let \(A\vect{x}=\vect{b}\) and \(A'\vect{x}=\vect{b}'\) be two systems of
linear equations with the same number of equations and the same number of
variables. If the augmented matrices \([A|\vect{b}]\) and \([A'|\vect{b}']\) are
\emph{row equivalent}, then the two systems have the same solution set.
\end{theorem}
\begin{pf}
By row equivalence, there exist elementary matrices \(E_1,\dotsc,E_k\) such
that
\[
E_k\dotsb E_1[A|\vect{b}]=[A'|\vect{b}'].
\]
By block multiplication, we can write
\[
[E_k\dotsb E_1A|E_k\dotsb E_1\vect{b}]=[A'|\vect{b}'],
\]
which implies that \(A'=E_{k}\dotsb E_1A\) and \(b'=E_{r}\dotsb E_1b\).

To prove that the two systems have the same solution set, it suffices to prove
the logical equivalence \(A\vect{v}=\vect{b}\iff A'\vect{v}=\vect{b}'\).

``\(\Rightarrow\)'': Assume that \(A\vect{v}=\vect{b}\). Then, multiplying
\(E_k\dotsb E_1\) on both sides gives
\[
E_{k}\dotsb E_1A\vect{v}=E_{k}\dotsb E_1\vect{b},
\]
thus \(A'\vect{v}=\vect{b}'\).

``\(\Leftarrow\)'': Assume that \(A'\vect{v}=\vect{b}'\). Then, we have
\[
E_{k}\dotsb E_1A\vect{v}=E_{k}\dotsb E_1\vect{b}.
\]
Since elementary matrices are invertible, we have
\[
E_{1}^{-1}\dotsb E_{k}^{-1}E_{k}\dotsb E_1A\vect{v}=E_{1}^{-1}\dotsb E_{k}^{-1}E_{k}\dotsb E_1\vect{b},
\]
which implies \(A\vect{v}=\vect{b}\).
\end{pf}

\item A corollary \cref{thm:row-equiv-same-sol} is the following \emph{column
correspondence property}. Loosely, it suggests that there are identical ``linear
relationships'' between \emph{columns} for two row equivalent matrices.

\begin{corollary}[Column correspondence property]
\label{cor:col-correspond}
Let \(A\) and \(B\) be two \(m\times n\) matrices that are row equivalent.
Write \(A=\mqty[\vect{a}_{1}&\cdots&\vect{a}_{n}]\) and
\(B=\mqty[\vect{b}_{1}&\cdots&\vect{b}_{n}]\). Then,
\[
c_1\vect{a}_{1}+\dotsb+c_n\vect{a}_{n}=\vect{0}\iff
c_1\vect{b}_{1}+\dotsb+c_n\vect{b}_{n}=\vect{0}
\]
where \(c_1,\dotsc,c_n\) are some constants.
\end{corollary}
\begin{pf}
It follows from \cref{thm:row-equiv-same-sol} with
\(\vect{b}=\vect{b}'=\vect{0}\) and \(B=A'\).
\end{pf}

\item Given a matrix \(A\), the \defn{RREF of \(A\)} is a matrix in RREF that
is row equivalent to \(A\).

Due to \cref{thm:row-equiv-same-sol}, our main goal here is to find a RREF of
the given augmented matrix, so that solving the underlying system is easy
afterwards. \emph{Gaussian elimination} provides a method for finding it
systematically.

\item The \defn{Gaussian elimination} is a standard procedure for obtaining
RREF of a matrix as follows.
\begin{enumerate}[label=(\arabic*)]
\item Identify the leftmost nonzero column.
\item Make the top entry of this column (called the \defn{pivot position})
nonzero by a type I ERO (if needed).
\item Make the entries below the pivot position zero by type III EROs (if
needed).
\item Ignore the row containing the pivot position and all rows above, and
focus on the submatrix remaining.
\item Repeat (1)--(4) as long as there is a nonzero row in the remained
submatrix.
\item Make the last pivot position \(1\) by a type II ERO (if needed).
\item Make all entries in the same column as this pivot position \(0\) by type
III EROs (if needed).
\item Repeat (6)--(7) with the rows above, until all rows have been handled.
\end{enumerate}

\newpage
\item Example of Gaussian elimination: Consider an augmented matrix 
\(\displaystyle A=\left[\begin{array}{@{}ccc|c@{}}
0&0&1&2\\
2&2&4&6\\
1&0&1&1
\end{array}\right]\).
\begin{align*}
&\left[\begin{array}{@{}ccc|c@{}}
0&0&1&2\\
2&2&4&6\\
1&0&1&1
\end{array}\right]\\
(1)\qquad&\left[\begin{array}{@{}ccc|c@{}}
\blc{0}&0&1&2\\
\blc{2}&2&4&6\\
\blc{1}&0&1&1
\end{array}\right]\\
(2)\qquad\overset{\vect{r}_{1}\leftrightarrow\vect{r}_{2}}{\longrightarrow}&
\left[\begin{array}{@{}ccc|c@{}}
\gc{2}&\gc{2}&\gc{4}&\gc{6}\\
\gc{0}&\gc{0}&\gc{1}&\gc{2}\\
1&0&1&1
\end{array}\right]\\
(3)\qquad\overset{-\frac{1}{2}\vect{r}_{1}+\vect{r}_{3}\to\vect{r}_{3}}{\longrightarrow}&
\left[\begin{array}{@{}ccc|c@{}}
\rc{2}&2&4&6\\
0&0&1&2\\
\vc{0}&\vc{-1}&\vc{-1}&\vc{-2}
\end{array}\right]\\
(4)\qquad&
\left[\begin{array}{@{}ccc|c@{}}
\gyc{2}&\gyc{2}&\gyc{4}&\gyc{6}\\
0&0&1&2\\
0&-1&-1&-2
\end{array}\right]\\
(1)\qquad&
\left[\begin{array}{@{}ccc|c@{}}
\gyc{2}&\gyc{2}&\gyc{4}&\gyc{6}\\
0&\blc{0}&1&2\\
0&\blc{-1}&-1&-2
\end{array}\right]\\
(2)\qquad\overset{\vect{r}_{2}\leftrightarrow\vect{r}_{3}}{\longrightarrow}&
\left[\begin{array}{@{}ccc|c@{}}
\gyc{2}&\gyc{2}&\gyc{4}&\gyc{6}\\
\gc{0}&\gc{-1}&\gc{-1}&\gc{-2}\\
\gc{0}&\gc{0}&\gc{1}&\gc{2}\\
\end{array}\right]\\
(3)\qquad&
\left[\begin{array}{@{}ccc|c@{}}
\gyc{2}&\gyc{2}&\gyc{4}&\gyc{6}\\
0&\rc{-1}&-1&-2\\
0&0&1&2\\
\end{array}\right]\text{\cmark}\\
(4)\qquad&
\left[\begin{array}{@{}ccc|c@{}}
\gyc{2}&\gyc{2}&\gyc{4}&\gyc{6}\\
\gyc{0}&\gyc{-1}&\gyc{-1}&\gyc{-2}\\
0&0&1&2\\
\end{array}\right]\\
(1)\qquad&
\left[\begin{array}{@{}ccc|c@{}}
\gyc{2}&\gyc{2}&\gyc{4}&\gyc{6}\\
\gyc{0}&\gyc{-1}&\gyc{-1}&\gyc{-2}\\
0&0&\blc{1}&2\\
\end{array}\right]\\
(4)\qquad&
\left[\begin{array}{@{}ccc|c@{}}
\gyc{2}&\gyc{2}&\gyc{4}&\gyc{6}\\
\gyc{0}&\gyc{-1}&\gyc{-1}&\gyc{-2}\\
\gyc{0}&\gyc{0}&\gyc{1}&\gyc{2}\\
\end{array}\right]\\
\end{align*}

\begin{align*}
(6)\qquad&
\left[\begin{array}{@{}ccc|c@{}}
2&2&4&6\\
0&-1&-1&-2\\
0&0&\rc{1}&2\\
\end{array}\right]\text{\cmark}\\
(7)\qquad\overset{\stackon{\(\vect{r}_{3}+\vect{r}_{2}\to\vect{r}_{2}\)}
{\(-4\vect{r}_{3}+\vect{r}_{1}\to\vect{r}_{1}\)}}{\longrightarrow}&
\left[\begin{array}{@{}ccc|c@{}}
\vc{2}&\vc{2}&\vc{0}&\vc{-2}\\
\vc{0}&\vc{-1}&\vc{0}&\vc{0}\\
0&0&\rc{1}&2\\
\end{array}\right]\\
&
\left[\begin{array}{@{}ccc|c@{}}
2&2&0&-2\\
0&\rc{-1}&0&0\\
0&0&1&2\\
\end{array}\right]\\
(6)\qquad\overset{-\vect{r}_{2}\to\vect{r}_{2}}{\longrightarrow}&\left[\begin{array}{@{}ccc|c@{}}
2&2&0&-2\\
\brc{0}&\brc{1}&\brc{0}&\brc{0}\\
0&0&1&2\\
\end{array}\right]\\
(7)\qquad\overset{-2\vect{r}_{2}+\vect{r}_{3}\to\vect{r}_{3}}{\longrightarrow}&\left[\begin{array}{@{}ccc|c@{}}
\vc{2}&\vc{0}&\vc{0}&\vc{-2}\\
0&1&0&0\\
0&0&1&2\\
\end{array}\right]\\
&\left[\begin{array}{@{}ccc|c@{}}
\rc{2}&0&0&-2\\
0&1&0&0\\
0&0&1&2\\
\end{array}\right]\\
(6)\qquad\overset{\frac{1}{2}\vect{r}_{1}\to\vect{r}_{1}}{\longrightarrow}
&\left[\begin{array}{@{}ccc|c@{}}
\brc{1}&\brc{0}&\brc{0}&\brc{-1}\\
0&1&0&0\\
0&0&1&2\\
\end{array}\right].
\end{align*}
Hence, the RREF of \(A\) is given by:
\[
\left[\begin{array}{@{}ccc|c@{}}
1&0&0&-1\\
0&1&0&0\\
0&0&1&2\\
\end{array}\right].
\]
This suggests that the solution to the underlying system is given by
\(x_1=-2\), \(x_2=0\), and \(x_3=2\).

\item Gaussian elimination ensures the \emph{existence} of RREF of any given
matrix \(A\), since this standard procedure can be applied to any matrix \(A\)
to obtain a matrix in RREF that is row equivalent to \(A\). Expressing this
fact using elementary matrices, we can say that there always exist elementary
matrices \(E_1,\dotsc,E_k\) such that
\[
R=E_k\dotsb E_1A
\]
where \(R\) is a RREF of \(A\). Naturally, one would then ask whether such RREF
of \(A\) is \emph{unique}. It turns out that it is indeed unique:

\begin{theorem}
\label{thm:rref-unique}
Every \(m\times n\) matrix \(A\) has a unique RREF.
\end{theorem}
\begin{pf}
(Sketch) The existence of RREF of \(A\) is guaranteed by Gaussian elimination.
So it suffices to prove the uniqueness part.

We may prove this by induction on \(n\) (the number of columns). The
result holds when \(n=1\), since there are only two possible \(m\times 1\)
matrices in RREF:
\[
\mqty[1\\ 0\\\vdots\\ 0]\qqtext{and}\mqty[0\\ 0\\\vdots\\ 0].
\]
Note that the leading one must be at the first row for the former matrix, so
that all zero rows lie at the bottom of the matrix. It is then not hard to see
that the unique RREF of any nonzero \(m\times 1\) matrix is the former one, and
the unique RREF of the zero \(m\times 1\) matrix is the latter one (itself).

Now, assume the result holds when \(n=k\), for some \(k\in\N\). Let \(R_1\) and
\(R_2\) be two RREFs of \(A\). Let \(A'\), \(R_1'\), and \(R_2'\) denote the
matrix obtained by removing the last column of \(A\), \(R_1\), and \(R_2\)
respectively.

\textbf{Claim:} \(R_1'\) and \(R_2'\) are two RREFs of \(A'\).

\begin{pf}
First of all, it is easy to see that \(R_1'\) and \(R_2'\) are both row equivalent
to \(A'\), since \(R_1'\) and \(R_2'\) can be obtained by the respective EROs
performed to get \(R_1\) and \(R_2\) from \(A\).

It then suffices to show that \(R_1'\) and \(R_2'\) are in RREF. WLOG, we focus
on \(R_1'\). When \(R\) is in RREF, we have the following cases.
\begin{itemize}
\item Case 1: All the leading ones are \emph{not} at the last column of \(R\).

In this case, removing the last column still retains all leading ones, and
every zero row (if exists) still remains as a zero row. It is not hard to see
that the leading ones still satisfy with the requirements imposed by RREF (as
they are not ``affected'' by the removal).

\item Case 2: The last column of \(R\) contains a leading one.

Since \(R\) is in RREF, every row below the row with leading one at the last
column (if exists) must be zero row. So, after removing the last column, the
original row with leading one at the last column becomes a zero row (and of
course every original zero row, if exist, is still a zero row). Similarly, the
other leading ones are not ``affected'' by the removal, and so still satisfy
the requirements from RREF.
\end{itemize}
\end{pf}

Then, the induction hypothesis implies that \(R_1'=R_2'\). Next, we can show
that the last columns of \(R_1\) and \(R_2\) are also the same by using
\cref{cor:col-correspond}.
\end{pf}

\item The RREF of a square matrix can provide information on the invertibility
of the square matrix, as suggested by the following result.

\begin{proposition}
\label{prp:matx-inv-iff-rref-inv}
Let \(A\) be an \(n\times n\) matrix. Then \(A\) is invertible iff the RREF of
\(A\) is invertible.
\end{proposition}
\begin{pf}
Let \(R\) be the RREF of \(A\). Then, we can write
\[
R=E_k\dotsb E_1A
\]
for some elementary matrices \(E_1,\dotsc,E_k\). Now we consider
\emph{determinant}. Since elementary matrices are invertible, \(\det E_i\ne 0\)
for any \(i=1,\dotsc,k\). Thus, by multiplicativity of determinant we have
\[
\det R=(\text{nonzero constant})\times \det A.
\]
This then suggests that \(\det A=0\) iff \(\det R=0\), so the desired result
follows.
\end{pf}

\item \label{it:solve-sle-algo} Since Gaussian elimination allows us to convert
an arbitrary matrix to its RREF, it gives us a general and systematic algorithm
for solving a system of linear equations (\emph{assuming that the system has a
solution}) as follows.

\begin{enumerate}
\item Given a system of linear equations \(A\vect{x}=\vect{b}\), consider its
associated augmented matrix \([A|\vect{b}]\). Using Gaussian elimination, we
can find its RREF, denoted by \([A'|\vect{b}']\). \begin{note}
By the claim in the proof of \cref{thm:rref-unique}, \(A'\) is indeed the RREF
of \(A\).
\end{note}

\item Every variable in the system corresponding to a leading entry of
\([A'|\vect{b}']\) is called a \defn{leading variable} (or \defn{basic
variable}). Every other variable in the system is known as \defn{free
variable}.
\item Set the free variables (if any) as arbitrary values, and then obtain the
solutions for leading variables based on the RREF \([A'|\vect{b}']\), possibly
in terms of the free variables.
\end{enumerate}
\begin{note}
The solution set obtained is sometimes called the \defn{general solution} of
the system (especially when some free variables are involved).
\end{note}
\end{enumerate}
\subsection{Number of Solutions for a System of Linear Equations}
\label{subsect:sys-sol-num}
\begin{enumerate}
\item In \labelcref{it:solve-sle-algo}, we have assumed the system has a
solution. But in general, a system of linear equations may have no solution,
e.g.
\[
\systeme{0x_1+0x_2=1,
x_1+x_2=2}
\]
has no solution. So, we are interested in finding some ways to determine the
number of solutions for a system.

\item A system of linear equations is \defn{consistent} if it has at least one
solution. Otherwise, the system is called \defn{inconsistent}. The following
theorem provides a criterion of (in)consistency, using the notion of RREF.

\begin{theorem}
\label{thm:consist-criterion}
A system of linear equations \(A\vect{x}=\vect{b}\) is \emph{inconsistent} iff
the RREF of the augmented matrix \([A|\vect{b}]\) has a leading one in the last
\emph{column}.
\end{theorem}
\begin{pf}
``\(\Leftarrow\)'': When the RREF has a leading one in the last column, it
means that a certain row of the RREF is
\[
\mqty[0&0&\cdots&0|1],
\]
which implies that there is no solution.

``\(\Rightarrow\)'': We prove by contrapositive. Assume that the RREF of
\([A|\vect{b}]\) does not have any leading one in the last column. Then, for
every row with a leading one, we can obtain an expression for the corresponding
leading variable (possibly in terms of free variables).\footnote{Note that such
expression would not contain other leading variables since by definition of
RREF, all other entries in the same column as leading one are zero.} After
setting the free variables (if any) as arbitrary values, we would obtain a
solution for the system, so the system is consistent.
\end{pf}

\item When we know the exact distribution of leading ones, we can say even
more.
\begin{corollary}
\label{cor:unique-criterion}
A system of linear equations \(A\vect{x}=\vect{b}\) has a \emph{unique}
solution iff the RREF of the augmented matrix \([A|\vect{b}]\) has a leading
one in each column \emph{except the last one}.
\end{corollary}
\begin{pf}
``\(\Leftarrow\)'': Assume that the RREF has a leading one in each column
except the last one. Then by \cref{thm:consist-criterion}, the system is
consistent. Furthermore, by having a leading one in each column except the
last one, there must be no free variable, so the solution is unique.

``\(\Rightarrow\)'': Assume that the system has a unique solution. Then, the
system is consistent, thus there is no leading one in the last column by
\cref{thm:consist-criterion}. Furthermore, due to the uniqueness of solution,
the system cannot possibly have any free variable. Hence, every other column
must have a leading one.
\end{pf}

\item After deducing that a system is consistent, we would like to know the
\emph{number} of solutions it has. Of course when the coefficient matrix is
invertible or when the RREF of the augmented matrix has a leading one in each
column except the last one, the system would have a unique solution, by
\cref{prp:inv-sle-unique} or \cref{cor:unique-criterion} respectively.

But this is not necessarily the case, and it is possible to have
\emph{infinitely many} solutions. For example, the system
\[
\systeme{x_1+x_2=0,
2x_1+2x_2=0}
\]
has infinitely many solutions. The solution set is \(\{(t,-t):t\in\R\}\), which
is infinite.

\item To study the case with infinitely many solutions, an useful notion is
\emph{homogeneity}.  A system of linear equations \(A\vect{x}=\vect{b}\) is
called \defn{homogeneous} if \(\vect{b}=\vect{0}\). Note that a homogeneous
system is \emph{always} consistent since the zero vector \(\vect{x}=\vect{0}\)
is always a solution (known as \defn{trivial solution}). Every other solution
of a homogeneous system, if exists, is called \defn{non-trivial solution}.

\item The following result provides a sufficient condition for a homogeneous
system to have infinitely many solutions.

\begin{proposition}
\label{prp:homo-var-more-than-eqs-inf}
Let \(A\vect{x}=\vect{0}_n\) be a homogeneous system of \(m\) linear equations in
\(n\) variables. If the number of variables (\(n\)) exceeds that of equations
(\(m\)), then the system has infinitely many solutions.
\end{proposition}
\begin{pf}
Since the augmented matrix of the system is \([A|\vect{0}_{n}]\), its RREF
would be of the form \([A'|\vect{0}_{n}]\) for some \(m\times n\) matrix \(A\).
As \(n>m\) and \(A'\) can have at most \(m\) leading entries, there must be at
least one free variable, which must lead to infinitely many solutions (since
the system is consistent).
\end{pf}

\item Utilizing \cref{prp:homo-var-more-than-eqs-inf}, we can show a more
general result that applies to any consistent system.

\begin{proposition}
\label{prp:var-more-than-eqs-sol-set}
Let \(A\vect{x}=\vect{b}\) be a system of \(m\) linear equations in \(n\)
variables. Suppose that \(n>m\) and the system is consistent. Let
\(\vect{x}_p\) be a \emph{particular solution} of the system, i.e., it
satisfies \(A\vect{x}_p=\vect{b}\). Then, the solution set for the system
\(A\vect{x}=\vect{b}\) is
\[
\{\vect{x}_p+\vect{y}\in\R^m:A\vect{y}=\vect{0}\}.
\]
\end{proposition}
\begin{note}
Since there are infinitely many \(\vect{y}\) satisfying \(A\vect{y}=\vect{0}\)
in such case by \cref{prp:homo-var-more-than-eqs-inf}, the solution set is
infinite, thus the system \(A\vect{x}=\vect{b}\) has infinitely many solutions
as well.
\end{note}

\begin{pf}
Let \(S\) be the solution set for the system \(A\vect{x}=\vect{b}\), i.e.,
\(S=\{\vect{x}\in\R^m:A\vect{x}=\vect{b}\}\). We shall prove the two subset
inclusions separately.

\underline{\(S\supseteq \{\vect{x}_p+\vect{y}\in\R^m:A\vect{y}=\vect{0}\}\)}:
Fix any \(\vect{x}'\in\{\vect{x}_p+\vect{y}\in\R^m:A\vect{y}=\vect{0}\}\). Then,
\(\vect{x}'=\vect{x}_p+\vect{y}\) for some \(\vect{y}\) satisfying
\(A\vect{y}=\vect{0}\). Hence,
\[
A\vect{x}'=A(\vect{x}_p+\vect{y})=A\vect{x}_p+A\vect{y}=\vect{b}+\vect{0}=\vect{b},
\]
which implies that \(\vect{x}'\in S\).

\underline{\(S\subseteq \{\vect{x}_p+\vect{y}\in\R^m:A\vect{y}=\vect{0}\}\)}:
Fix any \(\vect{x}'\in S\). Then, we have \(A\vect{x}'=\vect{b}\), thus
\(A\blc{(\vect{x}'-\vect{x}_p)}=\vect{b}-\vect{b}=\vect{0}\). Hence,
\[
\vect{x}'=\vect{x}_p+\blc{(\vect{x}'-\vect{x}_p)}
\in\{\vect{x}_p+\vect{y}\in\R^m:A\vect{y}=\vect{0}\}.
\]
\end{pf}

\item So far we have seen systems with unique solution or infinitely many
solutions. A natural follow-up question is then whether it is possible to have
more than one solution but still \emph{finitely} many solutions. It turns out
that this is impossible, as suggested by the following result.

\begin{proposition}
\label{prp:sle-more-than-one-sol-inf}
If a system of linear equations \(A\vect{x}=\vect{b}\) has more than one
solution, then it must have infinitely many solutions.
\end{proposition}
\begin{pf}
Assume that \(A\vect{x}=\vect{b}\) has more than one solution. Let
\(\vect{x}_1\) and \(\vect{x}_2\) be two such solutions. Then, we have
\(A\vect{x}_1=\vect{b}\) and \(A\vect{x}_2=\vect{b}\). It follows that
\(A(\vect{x}_1-\vect{x}_2)=\vect{b}-\vect{b}=\vect{0}\). Thus, \emph{for any}
\(t\in\R\), we have
\[
A[\vect{x}_1+t(\vect{x_1}-\vect{x}_2)]
=\vect{0}+t\vect{0}
=\vect{0},
\]
so \(\vect{x}_1+t(\vect{x}_1-\vect{x}_2)\) is also a solution of the system.
This then suggests that the system has infinitely many solutions.
\end{pf}

\item The next result concerns with a homogeneous system with the same numbers
of variables and equations.

\begin{proposition}
\label{prp:homo-same-var-eqs-equiv}
Let \(A\) be an \(n\times n\) matrix. Then, the following statements are
equivalent.
\begin{enumerate}
\item The system \(A\vect{x}=\vect{0}\) has a non-trivial solution.
\item The system \(A\vect{x}=\vect{0}\) has infinitely many solutions.
\item \(A\) is \emph{not} invertible.
\end{enumerate}
\end{proposition}
\begin{pf}
Here we will prove that \(\text{(a)}\implies \text{(c)}\implies
\text{(b)}\implies \text{(a)}\).

\underline{\(\text{(a)}\implies \text{(c)}\)}: We prove by contrapositive.
Assume that \(A\) is invertible. Then by \cref{prp:inv-sle-unique}, the system
\(A\vect{x}=\vect{0}\) has a \emph{unique} solution. Since the trivial solution
must be a solution of the homogeneous system \(A\vect{x}=\vect{0}\), it follows
that it is the \emph{only} one, thus the system does \emph{not} have a
non-trivial solution.

\underline{\(\text{(c)}\implies\text{(b)}\)}: Assume that \(A\) is not
invertible. Then, the RREF of \(A\), denoted by \(R\), is also \emph{not} invertible
by \cref{prp:matx-inv-iff-rref-inv}. Note that the RREF of the augmented matrix
\([A|\vect{0}]\) can be written as \([R|\vect{0}]\). For the ``\(R\)'' part, it
follows from the claim in the proof of \cref{thm:rref-unique}. The RHS is still
\(\vect{0}\) since any ERO would have no impact on the column of zeros.

Now, by \cref{thm:rref-matx-inv-equiv}, since the \(n\times n\) matrix \(R\)
(in RREF) is not invertible, it must have \emph{less than} \(n\) leading ones.
This means there are less than \(n\) leading entries, implying the existence of
\emph{free variable}. It then follows that the system must have infinitely many
solutions, as a free variable can take arbitrary values.

\underline{\(\text{(b)}\implies\text{(a)}\)}: It is clear since having
infinitely many solutions implies that there are solutions other than the trivial
solution, i.e., there is a non-trivial solution.
\end{pf}
\end{enumerate}
\subsection{Finding Matrix Inverses via Elementary Row Operations}
\begin{enumerate}
\item It turns out that EROs are not only useful for solving system of linear
equations, but also helpful for \emph{finding matrix inverses}. Indeed, when
the sizes of the matrices involved are large, often this ERO-based approach is
more computationally feasible than the formula in \cref{cor:matrix-inv-fmla}.

\item Before introducing the methods of finding matrix inverses based on EROs,
we first prove some theoretical results. The first one provides yet another
criterion for invertibility.

\begin{proposition}
\label{prp:inv-iff-rref-identity}
Let \(A\) be an \(n\times n\) matrix. Then \(A\) is invertible iff the RREF of
\(A\) is \(I_n\).
\end{proposition}
\begin{pf}
Note that
\begin{align*}
&\text{\(A\) is invertible}\\
\iff&\text{RREF of \(A\) is invertible}&\text{(\cref{prp:matx-inv-iff-rref-inv})}\\
\iff&\text{RREF of \(A\) is \(I_n\)}&\text{(\cref{thm:rref-matx-inv-equiv})}.
\end{align*}
\end{pf}
\item The next result relates invertible matrix and elementary matrices.
\begin{corollary}
\label{cor:inv-matx-prod-elementary}
Every invertible matrix \(A\) is a product of elementary matrices.
\end{corollary}
\begin{pf}
Since \(A\) is invertible, by \cref{prp:inv-iff-rref-identity}, the RREF of
\(A\) is \(I_n\). Then, due to the row equivalence of \(A\) and \(I_n\) (which
is a \emph{symmetric} relation), we have
\[
A=E_r\dotsb E_1I_n=E_r\dotsb E_1
\]
for some elementary matrices \(E_1,\dotsc,E_r\), as desired.
\end{pf}
\item Now, we are ready to introduce the method of finding matrix inverse via
EROs.

\begin{theorem}
\label{thm:find-inv-ero}
Let \(A\) be an invertible \(n\times n\) matrix. Then the RREF of the \(n\times
2n\) augmented matrix \([A|I_n]\) is \([I_n|A^{-1}]\).
\end{theorem}
\begin{pf}
Assume that \(A\) is invertible. Then the RREF of \(A\) is \(I_n\) by
\cref{prp:inv-iff-rref-identity}. Hence, we have
\begin{equation}
\label{eq:inv-rref-identity}
E_r\dotsb E_1A=I_n
\end{equation}
for some elementary matrices \(E_1,\dotsc,E_r\).

Using matrix block multiplication, we have
\[
E_r\dotsb E_1[A|I_n]
=[E_r\dotsb E_1A|E_r\dotsb E_1I_n]
=[I_n|E_r\dotsb E_1]
\]
From \cref{eq:inv-rref-identity}, we can further write
\[
A=E_1^{-1}\dotsb E_r^{-1},
\]
so \(A^{-1}=E_r\dotsb E_1\) by \cref{prp:matrix-inv-prop}, as desired.
\end{pf}

\begin{remark}
\item Practically, to get from \([A|I_n]\) to \([I_n|A^{-1}]\), we typically use
\emph{Gaussian elimination} on the \(n\times 2n\) augmented matrix \([A|I_n]\).
(It is not hard to see that \([I_n|A^{-1}]\) is the RREF of \([A|I_n]\).)
\item This result also tells us how to express an invertible matrix as a
product of elementary matrices implicitly. 

We first perform Gaussian elimination on \([A|I_n]\) and record the EROs
performed. Then, by considering the reverse processes of the EROs
(corresponding to the inverses of elementary matrices, which are also
elementary), we can write
\[
A=E_1^{-1}\dotsb E_r^{-1}.
\]
\end{remark}
\end{enumerate}
