\section{Inner Product Spaces}
\label{sect:inner-product}
\subsection{Inner Product Spaces}
\begin{enumerate}
\item Given two vectors \(\vect{u}=\mqty[u_1&\cdots&u_n]^{T}\) and
\(\vect{v}=\mqty[v_1&\cdots&v_n]^{T}\) in \(\R^n\), their \emph{dot product} is
the sum of the products of the corresponding pairs of entries:
\[
\vect{u}\cdot\vect{v}=u_1v_1+\dotsb+u_nv_n.
\]
\item Here, we would like to generalize this notion to a general vector space
\(V\) (not necessarily \(\R^n\)). A difficulty is that vectors in \(V\) may not
have ``entries'' like the ones in \(\R^n\), so the above ``sum of products''
definition of dot product may not work anymore. We need an alternative approach.

\item Instead of explicitly defining how to compute the ``generalized dot
product'' in a vector space, we shall use an axiomatic approach, i.e., specify
some properties that need to be satisfied to qualify as a ``generalized dot
product'' (we call it \emph{inner product}). To achieve the ``generalization'',
those properties should be satisfied by the usual dot product in \(\R^n\).

\item Let \(V\) be a vector space. An \defn{inner product on \(V\)} is a
function \(\inner{\cdot}{\cdot}:V\times V\to\R\) satisfying the
following properties:
\begin{enumerate}[label={(\arabic*)}]
\item (preserving addition in the first argument) For any \(\vect{u},\vect{v},\vect{w}\in V\),
\(\inner{\vect{u}+\vect{v}}{\vect{w}}=\inner{\vect{u}}{\vect{w}}+\inner{\vect{v}}{\vect{w}}\).
\item (preserving scalar multiplication in the first argument) For any
\(\vect{u},\vect{v}\in V\) and \(c\in\R\),
\(\inner{c\vect{u}}{\vect{v}}=c\inner{\vect{u}}{\vect{v}}\).
\item (symmetry) For any \(\vect{u},\vect{v}\in V\),
\(\inner{\vect{u}}{\vect{v}}=\inner{\vect{v}}{\vect{u}}\).
\item (positive definiteness) For any \(\vect{v}\in V\),
\(\inner{\vect{v}}{\vect{v}}>0\) if \(\vect{v}\ne\vect{0}\).
\end{enumerate}
A vector space \(V\) equipped with an inner product \(\inner{\cdot}{\cdot}\) on
\(V\) is said to be an \defn{inner product space}, denoted by
\((V,\inner{\cdot} {\cdot})\). For convenience, often we just denote such inner
product space by \(V\) (and we shall use the notation \(\inner{\cdot}{\cdot} \)
to stand for the equipped inner product by default).

\item Based on these defining properties for inner product, we can readily
deduce some more extra properties:
\begin{proposition}
\label{prp:in-prod-prop}
Let \(V\) be an inner product space, and \(\vect{u},\vect{v},\vect{w}\) be any
vectors in \(V\). Then,
\begin{enumerate}
\item \(\inner{\vect{w}}{\vect{u}+\vect{v}}=\inner{\vect{w}}{\vect{u}}+\inner{\vect{w}}{\vect{v}}\).
\item For any \(c\in\R\),
\(\inner{\vect{u}}{c\vect{v}}=c\inner{\vect{u}}{\vect{v}}\).
\item \(\inner{\vect{v}}{\vect{0}}=\inner{\vect{0}}{\vect{v}}=0\).
\item \(\inner{\vect{v}}{\vect{v}}=0\) iff \(\vect{v}=\vect{0}\).
\end{enumerate}
\end{proposition}
\begin{pf}
\begin{enumerate}
\item Note that
\[
\inner{\vect{w}}{\vect{u}+\vect{v}}
\overset{\text{(3)}}{=}\inner{\vect{u}+\vect{v}}{\vect{w}}
\overset{\text{(1)}}{=}\inner{\vect{u}}{\vect{w}}+\inner{\vect{v}}{\vect{w}}
\overset{\text{(3)}}{=}\inner{\vect{w}}{\vect{u}}+\inner{\vect{w}}{\vect{v}}.
\]
\item Note that
\[
\inner{\vect{u}}{c\vect{v}}
\overset{\text{(3)}}{=}\inner{c\vect{v}}{\vect{u}}
\overset{\text{(2)}}{=}c\inner{\vect{v}}{\vect{u}}
\overset{\text{(3)}}{=}c\inner{\vect{u}}{\vect{v}}.
\]
\item Since \(0\vect{v}=\vect{0}\), we have
\[
\inner{\vect{0}}{\vect{v}}
=\inner{0\vect{v}}{\vect{v}}
\overset{\text{(3)}}{=}0\inner{\vect{v}}{\vect{v}}
=0,
\]
proving the second equality. The first equality follows from (3).

\item ``\(\Rightarrow\)'': We prove by contrapositive. Suppose
\(\vect{v}\ne\vect{0}\). Then, by (4), we have
\(\inner{\vect{v}}{\vect{v}}>0\).

``\(\Leftarrow\)'': Suppose \(\vect{v}=\vect{0}\). Then by (c),
\(\inner{\vect{v}}{\vect{v}}=0\).
\end{enumerate}
\end{pf}

\item Examples of inner product spaces:
\begin{itemize}
\item The vector space \(\R^n\) with \(\inner{\vect{u}}{\vect{v}}=u_1v_1+\dotsb+u_nv_n\)
for any \(\vect{u}=\mqty[u_1&\cdots&u_n]^{T}\) and
\(\vect{v}=\mqty[v_1&\cdots&v_n]^{T}\) in \(\R^n\) (the usual dot product).
This inner product is said to be the \defn{standard inner product} on \(\R^n\).

\begin{pf}
Fix any \(\vect{u}=\mqty[u_1&\cdots&u_n]^{T}\),
\(\vect{v}=\mqty[v_1&\cdots&v_n]^{T}\), and
\(\vect{w}=\mqty[w_1&\cdots&w_n]^{T}\) in \(\R^n\), and any \(c\in\R\). Now
check:

(1):
\begin{align*}
\inner{\vect{u}+\vect{v}}{\vect{w}}
&=(u_1+v_1)w_1+\dotsb+(u_n+v_n)w_n \\
&=(u_1w_1+\dotsb+u_nw_n)+(v_1w_1+\dotsb+v_nw_n) \\
&=\inner{\vect{u}}{\vect{w}}+\inner{\vect{v}}{\vect{w}}.
\end{align*}

(2):
\begin{align*}
\inner{c\vect{u}}{\vect{v}}
&=(cu_1)v_1+\dotsb+(cu_n)v_n \\
&=c(u_1v_1+\dotsb+u_nv_n) \\
&=c\inner{\vect{u}}{\vect{v}}.
\end{align*}

(3):

\begin{align*}
\inner{\vect{u}}{\vect{v}}
&=(cu_1)v_1+\dotsb+(cu_n)v_n \\
&=u_1v_1+\dotsb+u_nv_n \\
&=v_1u_1+\dotsb+v_nu_n \\
&=\inner{\vect{v}}{\vect{u}}.
\end{align*}

(4): When \(\vect{v}\ne\vect{0}\),
\[
\inner{\vect{v}}{\vect{v}}=v_1^2+\dotsb+v_n^2>0
\]
(since at least one of the \(n\) terms is positive, and all the \(n\) terms are
nonnegative).
\end{pf}
\item Let \((V,\inner{\cdot}{\cdot})\) be an inner product space, and \(r\) be
any \emph{positive} number. Define \(\inner{\cdot}{\cdot}':V\times V\to\R\) by
\(\inner{\vect{u}}{\vect{v}}'=r\inner{\vect{u}}{\vect{v}}\) for any
\(\vect{u},\vect{v}\in V\). Then \(\inner{\cdot}{\cdot}'\) is also an inner product on \(V\).

\begin{pf}
Fix any \(\vect{u},\vect{v},\vect{w}\in V\), and any \(c\in\R\). Now check:

(1):
\[
\inner{\vect{u}+\vect{v}}{\vect{w}}'
=r\inner{\vect{u}+\vect{v}}{\vect{w}}
=r(\inner{\vect{u}}{\vect{w}}+\inner{\vect{v}}{\vect{w}})
=\inner{\vect{u}}{\vect{w}}'+\inner{\vect{v}}{\vect{w}}'.
\]

(2):
\[
\inner{c\vect{u}}{\vect{v}}'
=r\inner{c\vect{u}}{\vect{v}}
=cr\inner{\vect{u}}{\vect{v}}
=c\inner{\vect{u}}{\vect{v}}'.
\]

(3):

\[
\inner{\vect{u}}{\vect{v}}'
=r\inner{\vect{u}}{\vect{v}}
=r\inner{\vect{v}}{\vect{u}}
=\inner{\vect{v}}{\vect{u}}'.
\]

(4): When \(\vect{v}\ne\vect{0}\),
\[
\inner{\vect{v}}{\vect{v}}'=r\underbrace{\inner{\vect{v}}{\vect{v}}}_{>0}
>0.
\]
\end{pf}

\begin{note}
The positivity of \(r\) is only needed for proving (4).
\end{note}


\item Let \(T:V\to W\) be an \emph{injective} linear transformation, and let
\(\inner{\cdot}{\cdot}_W\) be an inner product on \(W\). For any
\(\vect{x},\vect{y}\in V\), define
\(\inner{\vect{x}}{\vect{y}}_V=\inner{T(\vect{x})}{T(\vect{y})}_W\). Then,
\(\inner{\cdot}{\cdot}_V\) is an inner product on \(V\).

\begin{pf}
Fix any \(\vect{x},\vect{y},\vect{z}\in V\), and any \(c\in\R\). Now check:

(1):
\begin{align*}
\inner{\vect{x}+\vect{y}}{\vect{z}}_{V}
&=\inner{T(\vect{x}+\vect{y})}{T(\vect{z})}_{W} \\
&=\inner{T(\vect{x})+T(\vect{y})}{T(\vect{z})}_{W} \\
&=\inner{T(\vect{x})}{T(\vect{y})}_{W} + \inner{T(\vect{x})}{T(\vect{z})}_{W} \\
&=\inner{\vect{x}}{\vect{y}}_V+\inner{\vect{x}}{\vect{z}}_{V}.
\end{align*}

(2):
\begin{align*}
\inner{c\vect{x}}{\vect{y}}_{V}
&=\inner{T(c\vect{x})}{T(\vect{y})}_{W} \\
&=\inner{cT(\vect{x})}{T(\vect{y})}_{W} \\
&=c\inner{T(\vect{x})}{T(\vect{y})}_{W}\\
&=c\inner{\vect{x}}{\vect{y}}_V.
\end{align*}

(3):

\begin{align*}
\inner{\vect{x}}{\vect{y}}_{V}
&=\inner{T(\vect{x})}{T(\vect{y})}_{W}\\
&=\inner{T(\vect{y})}{T(\vect{x})}_{W}\\
&=\inner{\vect{y}}{\vect{x}}_{V}.
\end{align*}

(4): When \(\vect{v}\ne\vect{0}\), since \(T\) is injective, we must have
\(T(\vect{v})\ne\vect{0}=T(\vect{0})\). Hence,
\[
\inner{\vect{v}}{\vect{v}}_{V}=\inner{T(\vect{v})}{T(\vect{v})}{W}
>0.
\]
\end{pf}

\begin{note}
The injectivity of \(T\) is only needed for proving (4).
\end{note}
\end{itemize}

\item \label{it:std-inner-prod-matx-trans}
Standard inner products naturally induce the occurrence of matrix transposes.
Let \(A\) be an \(m\times n\) matrix, and \(\vect{u}\in\R^n,\vect{v}\in\R^m\)
be any column vectors. Then,
\(\inner{A\vect{u}}{\vect{v}}=\inner{\vect{u}}{A^{T}\vect{v}}\), where
\(\inner{\cdot}{\cdot}\) is the standard inner product on \(\R^m\).

\begin{pf}
For the standard inner product \(\inner{\cdot}{\cdot}\), we have
\(\inner{\vect{u}}{\vect{v}}=\vect{u}^{T}\vect{v}\) for any
\(\vect{u},\vect{v}\in\R^m\). Hence,
\[
\inner{A\vect{u}}{\vect{v}}
=(A\vect{u})^{T}\vect{v}
=\vect{u}^{T}A^{T}\vect{v}
=\vect{u}^{T}(A^{T}\vect{v})
=\inner{\vect{u}}{A^{T}\vect{v}}.
\]
\end{pf}

\item After exploring the theoretical properties of inner products, we turn our
attention to their geometric interpretation. In fact, apart from being a
``generalized dot product'', inner product also serves as a generalization to
the geometric concept of \emph{length}.

\item Let \((V,\inner{\cdot}{\cdot})\) be an inner product space. For
any \(\vect{v}\in V\), the \defn{length} or \defn{norm} of \(\vect{v}\) is
\(\norm{\vect{v}}=\sqrt{\inner{\vect{v}}{\vect{v}}}\).

\item Example: Consider \(\R^n\) equipped with the standard inner product.
Then, the norm of any vector \(\vect{v}=\mqty[v_1&\cdots&v_n]^{T}\in\R^n\) is
\(\norm{\vect{v}}=\sqrt{v_1^2+\dotsb+v_n^2}\) (familiar?).

\item Next, we consider some properties of norm. The first one suggests that
norm-preserving linear transformation must be injective.

\begin{proposition}
\label{prp:lt-pre-norm-inj}
Let \(V\) be an inner product space and \(T:V\to V\) be a linear
transformation. Suppose that \(\norm{T(\vect{v})}=\norm{\vect{v}}\) for any
\(\vect{v}\in V\) (so \(T\) \emph{preserves the norm}). Then, \(T\) is injective.
\end{proposition}
\begin{pf}
By \Cref{prp:inj-iff-null-sp-only-zero}, it suffices to prove that
\(\nul{T}\subseteq \{\vect{0}\}\). Suppose \(T(\vect{v})=\vect{0}\). Then,
\(\norm{\vect{v}}=\norm{T(\vect{v})}=\norm{\vect{0}}=0\), which implies that
\(\inner{\vect{v}}{\vect{v}}=0\), thus \(\vect{v}=\vect{0}\) by
\Cref{prp:in-prod-prop}.
\end{pf}

\item The result below gives us more properties about norm.

\begin{proposition}
\label{prp:norm-prop}
Let \(V\) be an inner product space. Fix any \(\vect{u},\vect{v}\in V\) and
\(c\in\R\). Then,
\begin{enumerate}
\item (norm after scaling) \(\norm{c\vect{v}}=|c|\norm{\vect{v}}\).
\item (only zero vector has zero norm) \(\norm{\vect{v}}=0\) iff \(\vect{v}=\vect{0}\).
\item (Cauchy-Schwartz inequality) \(\inner{\vect{u}}{\vect{v}}\le\norm{\vect{u}}\norm{\vect{v}}\).
\item (triangle inequality) \(\norm{\vect{u}+\vect{v}}\le\norm{\vect{u}}+\norm{\vect{v}}\).
\end{enumerate}
\end{proposition}
\begin{pf}
\begin{enumerate}
\item Note that
\[
\|c\vect{v}\|=\sqrt{\inner{c\vect{v}}{c\vect{v}}}
=\sqrt{c^2\inner{\vect{v}}{\vect{v}}}
=|c|\sqrt{\inner{\vect{v}}{\vect{v}}}
=|c|\norm{\vect{v}}.
\]
\item Note that
\[
\norm{\vect{v}}=0
\iff \inner{\vect{v}}{\vect{v}}=0
\overset{\text{\Cref{prp:in-prod-prop}}}{\iff}\vect{v}=\vect{0}.
\]
\item For any \(c\in\R\), we have
\begin{align*}
0&\le\inner{\vect{u}-c\vect{v}}{\vect{u}-c\vect{v}} \\
&=\inner{\vect{u}}{\vect{u}-c\vect{v}}
-c\inner{\vect{v}}{\vect{u}-c\vect{v}}
 \\
&=\inner{\vect{u}}{\vect{u}}-c\inner{\vect{u}}{\vect{v}}-c\inner{
\vect{v}}{\vect{u}}+c^{2}\inner{\vect{v}}{\vect{v}} \\
&=\inner{\vect{u}}{\vect{u}}-2c\inner{\vect{u}}{\vect{v}}+c^{2}\inner{\vect{v}}{\vect{v}}.
\end{align*}
Then, we take \(c=\inner{\vect{u}}{\vect{v}}/\norm{\vect{v}}^{2}\)
(tricky!), so that the inequality can be simplified to
\begin{align*}
0&\le\norm{\vect{u}}^{2}
-2\frac{\inner{\vect{u}}{\vect{v}}}{\norm{\vect{v}}^{2}}\inner{\vect{u}}{\vect{v}}
+\qty(\frac{\inner{\vect{u}}{\vect{v}}}{\norm{\vect{v}}^{2}})^{2}\norm{\vect{v}}^{2} \\
&=\norm{\vect{u}}^{2}-\frac{\inner{\vect{u}}{\vect{v}}}{\norm{\vect{v}}^{2}},
\end{align*}
which implies that \(\inner{\vect{u}}{\vect{v}}^{2}\le\norm{\vect{u}}^{2}\norm{\vect{v}}^{2}\), hence \(\inner{\vect{u}}{\vect{v}}\le\norm{\vect{u}}\norm{\vect{v}}\).
\item We have
\begin{align*}
\norm{\vect{u}+\vect{v}}^{2}&=\inner{\vect{u}+\vect{v}}{\vect{u}+\vect{v}}\\
&=\inner{\vect{u}}{\vect{u}}+\inner{\vect{u}}{\vect{v}}
+\inner{\vect{v}}{\vect{u}}+\inner{\vect{v}}{\vect{v}} \\
&=\inner{\vect{u}}{\vect{u}}+2\inner{\vect{u}}{\vect{v}}
+\inner{\vect{v}}{\vect{v}} \\
&\le\norm{\vect{u}}^{2}+2\norm{\vect{u}}\norm{\vect{v}}+\norm{\vect{v}}^{2} &\text{(c)}\\
&=(\norm{\vect{u}}+\norm{\vect{v}})^{2}.
\end{align*}
\end{enumerate}
\end{pf}
\end{enumerate}
\subsection{Orthonormal Basis}
\label{subsect:orthonormal-basis}
\begin{enumerate}
\item Consider the inner product space \(\R^n\) equipped with the standard
inner product, and consider the standard basis
\(\{\vect{e}_1,\dotsc,\vect{e}_n\}\) for \(\R^n\).

As we have seen previously, it is nice to work with this basis. In fact, this
basis possesses some special properties that makes it an \emph{orthonormal
basis}. In \Cref{subsect:orthonormal-basis}, we shall investigate properties
enjoyed by orthonormal bases, and also discuss how to construct orthonormal
bases, in a general inner product space.

\item Let \(V\) be an inner product space. For any \(\vect{u},\vect{v}\in V\),
the vectors \(\vect{u}\) and \(\vect{v}\) are said to be \defn{orthogonal} if
\(\inner{\vect{u}}{\vect{v}}=0\).
\begin{note}
In the case where \(V=\R^2\text{ or }\R^3\) equipped with the standard inner
product, the geometrical meaning of being orthogonal is just being
perpendicular (excluding the zero vector in our consideration). So
orthogonality may be seen as a generalization to perpendicularity.
\end{note}

Also, we say that a subset \(S\) of \(V\) is \defn{orthogonal} if
\(\inner{\vect{x}}{\vect{y}}=0\) for any \emph{distinct} \(\vect{x},\vect{y}\in
S\), i.e., any two \emph{distinct} vectors in \(S\) are orthogonal.

\begin{remark}
\item We exclude the case where \(\vect{x}=\vect{y}\) since in
such case we have \(\inner{\vect{x}}{\vect{y}}=0\) iff
\(\vect{x}=\vect{0}\), by \Cref{prp:in-prod-prop}. So if we
did not include ``distinct'' in the definition, the only
orthogonal subset of \(V\) would be \(\{\vect{0}\}\)!
\item By definition, any singleton subset \(S\) of \(V\) is orthogonal since
there are not two distinct vectors in \(S\), so the condition is trivially true.
\end{remark}

\item Next, we will define orthonormality. Firstly, a vector \(\vect{v}\in V\)
is an \defn{unit vector} if \(\norm{\vect{v}}=1\). Then, a subset \(S\) of
\(V\) is called \defn{orthonormal} if (i) it is orthogonal and (ii) every
vector in \(S\) is an unit vector.

Examples and non-examples: (We consider the inner product space \(\R^2\)
equipped with standard inner product in the following.)
\begin{itemize}
\item The standard basis \(\{\vect{e}_1,\vect{e}_2\}\) is an orthonormal basis.
\item The standard basis \(\{2\vect{e}_1,2\vect{e}_2\}\) is orthogonal, but
\emph{not} orthonormal.
\item The basis \(\qty{\mqty[1\\ 1],\mqty[1\\ -1]}\) is orthogonal, but
\emph{not} orthonormal.
\item The basis \(\qty{\mqty[1/\sqrt{2}\\ 1/\sqrt{2}],\mqty[1/\sqrt{2}\\
-1/\sqrt{2}]}\) is an orthonormal basis.
\end{itemize}

\item Given an orthogonal subset \(S\) of \(V\) containing nonzero vectors, one
can always obtain an orthonormal subset from it by scaling the vectors inside
\(S\) such that they become unit vectors, i.e., \defn{normalizing} the vectors.
More specifically, the subset \(S'=\displaystyle
\qty{\frac{\vect{v}}{\norm{\vect{v}}}:\vect{v}\in S}\) is orthonormal.

\begin{pf}
Fix any distinct \(\vect{u}',\vect{v}'\in S'\). Then we have
\(\vect{u}'=\vect{u}/\norm{\vect{u}}\) and
\(\vect{v}'=\vect{v}/\norm{\vect{v}}\) where \(\vect{u},\vect{v}\in S\) are
distinct. Hence, due to the orthogonality of \(S\), we have
\[
\inner{\vect{u}'}{\vect{v}'}
=\frac{1}{\norm{\vect{u}}\norm{\vect{v}}}\inner{\vect{u}}{\vect{v}}
=\frac{0}{\norm{\vect{u}}\norm{\vect{v}}}
=0.
\]
This proves \(S'\) is orthogonal.

Next, consider any \(\vect{v}'\in S\). By \Cref{prp:norm-prop}, its norm is
\[
\norm{\vect{v}'}
=\qty|\frac{1}{\norm{\vect{v}}}|\norm{\vect{v}}
=\frac{1}{\norm{\vect{v}}}\norm{\vect{v}}
=1,
\]
hence \(S'\) is orthonormal.
\end{pf}

\item Now we discuss some properties of orthogonal sets. The first one is about
linear independence.

\begin{theorem}
\label{thm:orthog-imp-li}
Let \(V\) be an inner product space, and \(S\) be an orthogonal set of nonzero
vectors in \(V\). Then \(S\) is linearly independent.
\end{theorem}
\begin{intuition}
Some intuition can be gained by considering the case where \(V=\R^2\text{ or
}\R^3\). In such case, orthogonality is about perpendicularity and linear
independence is like being ``not parallel''. So, intuitively, being
perpendicular should imply being not parallel (but not converse).
\end{intuition}

\begin{pf}
Fix any \(n\in\N\) and any distinct \(\vect{v}_1,\dotsc,\vect{v}_n\in S\).
Suppose
\[
a_1\vect{v}_1+\dotsb+a_n\vect{v}_n=\vect{0},
\]
where \(a_1,\dotsc,a_n\) are scalars.

Now, fix any \(i=1,\dotsc,n\) and consider
\[
\inner{a_1\vect{v}_1+\dotsb+a_n\vect{v}_n}{\vect{v}_i}=a_i\inner{\vect{v}_i}{\vect{v}_i}.
\]
(All other terms vanish due to orthogonality of \(S\).)

On the other hand, we have
\[
\inner{a_1\vect{v}_1+\dotsb+a_n\vect{v}_n}{\vect{v}_i}
=\inner{\vect{0}}{\vect{v}_i}
=\vect{0}
\]
by \Cref{prp:in-prod-prop}.

This means that \(a_i\inner{\vect{v}_i}{\vect{v}_i}=\vect{0}\). But as
\(\vect{v}_i\) is nonzero (since \(S\) only contains nonzero vectors), we must
have \(\inner{\vect{v}_i}{\vect{v}_i}>0\). It follows that \(a_i=0\).

Applying this argument for every \(i=1,\dotsc,n\), we have
\(a_1=\dotsb=a_n=0\).
\end{pf}

The converse of \Cref{thm:orthog-imp-li} is not true. For example, consider the
inner product space \(\R^2\) equipped with standard inner product, and take
\(\vect{u}=\mqty[1\\ 1]\) and \(\vect{v}=\mqty[0\\ 1]\). Then
\(S=\{\vect{u},\vect{v}\}\) is linearly independent but \(S\) is not orthogonal
(since \(\inner{\vect{u}}{\vect{v}}=1\ne 0\).

\item Let \(V\) be a vector space and \(S=\{\vect{v}_1,\dotsc,\vect{v}_n\}\) be
a subset of \(V\). Given a vector \(\vect{v}\in \spn{S}\), expressing
\(\vect{v}\) as a linear combination of the vectors in \(S\) generally requires
us to solve a system of linear equation or use some other approach, which can
be cumbersome.

However, when we equip with \(V\) with an inner product, and if \(S\) is
orthogonal, then we would have a more convenient approach.

\begin{theorem}
\label{thm:orthog-lin-comb}
Let \(V\) be an inner product space, and \(S=\{\vect{v}_1,\dotsc,\vect{v}_n\}\)
be an orthogonal subset of \(V\) containing nonzero vectors. Then, for any
\(\vect{u}\in\spn{S}\), we have
\[
\vect{u}=\frac{\inner{\vect{u}}{\vect{v}_1}}{\norm{\vect{v}_1}^{2}}\vect{v}_1
+\dotsb+\frac{\inner{\vect{u}}{\vect{v}_n}}{\norm{\vect{v}_n}^{2}}\vect{v}_n.
\]
\end{theorem}
\begin{note}
Given \(\vect{u}\in\spn{S}\), \(\displaystyle
\frac{\inner{\vect{u}}{\vect{v}_i}}{\norm{\vect{v}_i}^{2}}\) can be found
directly by straightforward computation, for any \(i=1,\dotsc,n\). So we can
express \(\vect{u}\) as a linear combination of
\(\vect{v}_1,\dotsc,\vect{v}_n\) conveniently using this result.
\end{note}

\begin{pf}
Fix any \(\vect{u}\in\spn{S}\), and we can write
\[
\vect{u}=a_1\vect{v}_1+\dotsb+a_n\vect{v}_n
\]
for some \(a_1,\dotsc,a_n\in\R\).

Now, fix any \(i=1,\dotsc,n\), and consider
\[
\inner{\vect{u}}{\vect{v}_i}=\inner{a_1\vect{v}_1+\dotsb+a_n\vect{v}_n}{\vect{v}_i}
=a_i\inner{\vect{v}_i}{\vect{v}_i}.
\]
(All other terms vanish due to orthogonality of \(S\).) It follows that
\[
a_i=\frac{\inner{\vect{u}}{\vect{v}_i}}{\inner{\vect{v}_i}{\vect{v}_i}}
=\frac{\inner{\vect{u}}{\vect{v}_i}}{\norm{\vect{v}_i}^{2}}.
\]
The result then follows by applying this argument for every \(i=1,\dotsc,n\).
\end{pf}

\item After discussing some properties of orthogonal sets, we will introduce a
way to systematically construct orthonormal bases, known as \emph{Gram-Schmidt
process}. Let us start with the following result, which gives us a systematic
way to construct \emph{orthogonal} bases.

\begin{theorem}
\label{thm:construct-orthog-basis}
Let \(V\) be an inner product space and \(S=\{\vect{u}_1,\dotsc,\vect{u}_n\}\)
be a basis for \(V\). Let \(\vect{v}_1=\vect{u}_1\), and for any
\(j=2,\dotsc,n\), let
\[
\vect{v}_i=\vect{u}_i-\sum_{j=1}^{i-1}\frac{\inner{\vect{u}_i}{\vect{v}_j}}{\norm{\vect{v}_j}^{2}}\vect{v}_j.
\]
More explicitly, we let:
\begin{align*}
\vect{v}_1&=\vect{u}_1, \\
\vect{v}_2&=\vect{u}_2-\frac{\inner{\vect{u}_2}{\vect{v}_1}}{\norm{\vect{v}_1}^{2}}\vect{v}_1, \\
\vect{v}_3&=\vect{u}_3-\frac{\inner{\vect{u}_3}{\vect{v}_1}}{\norm{\vect{v}_1}^{2}}\vect{v}_1 
-\frac{\inner{\vect{u}_3}{\vect{v}_2}}{\norm{\vect{v}_2}^{2}}\vect{v}_2, \\
\vdots \\
\vect{v}_n&=\vect{u}_n-\frac{\inner{\vect{u}_n}{\vect{v}_1}}{\norm{\vect{v}_1}^{2}}\vect{v}_1
-\frac{\inner{\vect{u}_n}{\vect{v}_2}}{\norm{\vect{v}_2}^{2}}\vect{v}_2
-\dotsb
-\frac{\inner{\vect{u}_n}{\vect{v}_{n-1}}}{\norm{\vect{v}_{n-1}}^{2}}\vect{v}_{n-1}.
\end{align*}
Then, \(S'=\{\vect{v}_1,\dotsc,\vect{v}_n\}\) is an orthogonal basis for \(V\).
\end{theorem}
\begin{pf}
It suffices to show that \(S'\) is an orthogonal spanning set of \(V\), since
orthogonality implies linear independence by \Cref{thm:orthog-imp-li}, and
\(S'\) contains nonzero vectors as for any \(i=1,\dotsc,n\), \(\displaystyle
\vect{u}_i\ne\sum_{j=1}^{i-1}\frac{\inner{\vect{u}_i}{\vect{v}_j}}{\norm{\vect{v}_j}^{2}}\vect{v}_j\)
(since the latter is a linear combination of
\(\vect{u}_1,\dotsc,\vect{u}_{i-1}\), and if the equality held, it would
contradict the linear independence of \(S\)).

\underline{Spanning set}: For each \(i=1,\dotsc,n\), we can write
\begin{equation}
\label{eq:ui-expr-vi}
\vect{u}_i=\vc{\vect{v}_i}+\sum_{j=1}^{i-1}\frac{\inner{\vect{u}_i}{\vect{v}_j}}{\norm{\vect{v}_j}^{2}}\vc{\vect{v}_j},
\end{equation}
which shows that \(\vect{u}_i\) is a linear combination of
\(\vect{v}_1,\dotsc,\vect{v}_n\), thus \(\vect{u}_i\in\spn{S'}\).  Since this
holds for any \(i=1,\dotsc,n\), it follows that
\(V=\spn{\{\vect{u}_1,\dotsc,\vect{u}_n\}}\subseteq \spn{S'}\). On the other
hand, we readily have \(\spn{S'}\subseteq V\) since
\(\vect{v}_1,\dotsc,\vect{v}_n\in V\).

\underline{Orthogonal}: We use an induction argument. First, note that
\[
\inner{\vect{v}_2}{\vect{v}_1}
=\inner{\vect{u}_2-\frac{\inner{\vect{u}_2}{\vect{v}_1}}{\norm{\vect{v}_1}^{2}}\vect{v}_1}{\vect{v}_1}
=\inner{\vect{u}_2}{\vect{v}_1}-\frac{\inner{\vect{u}_2}{\vect{v}_1}}{\norm{\vect{v}_1}^{2}}\inner{\vect{v}_1}{\vect{v}_1}
=0.
\]
This shows \(\{\vect{v}_1,\vect{v}_2\}\) is orthogonal.

Now suppose for induction that \(\{\vect{v}_1,\dotsc,\vect{v}_k\}\) is
orthogonal for a \(k\in\{2,\dotsc,n-1\}\). Then, fix any
\(\ell\in\{1,\dotsc,k\}\), and consider
\begin{align*}
\inner{\vect{v}_{k+1}}{\vect{v}_{\ell}}
&=\inner{\vect{u}_{k+1}-\sum_{j=1}^{k}\frac{\inner{\vect{u}_{k+1}}{\vect{v}_j}}{\norm{\vect{v}_j}^{2}}\vect{v}_j}
{\vect{v}_{\ell}} \\
&=\inner{\vect{u}_{k+1}}{\vect{v}_{\ell}}
-\inner{\sum_{j=1}^{k}\frac{\inner{\vect{u}_{k+1}}{\vect{v}_j}}{\norm{\vect{v}_j}^{2}}\vect{v}_j}{\vect{v}_{\ell}} \\
&=0-\frac{\inner{\vect{u}_{k+1}}{\vect{v}_{\ell}}}{\norm{\vect{v}_{\ell}}^{2}}\inner{\vect{v}_{\ell}}{\vect{v}_{\ell}}&\text{(by induction hypothesis)}\\
&=-\inner{\vect{u}_{k+1}}{\vect{v}_{\ell}} \\
&=0&\text{(\(\vect{v}_{\ell}\) is a linear combination of \(\vect{u}_1,\dotsc,\vect{u}_{\ell}\))}.
\end{align*}
It follows that \(\{\vect{v}_1,\dotsc,\vect{v}_{k+1}\}\) is orthogonal. Hence
\(\{\vect{v}_1,\dotsc,\vect{v}_n\}\) is orthogonal.
\end{pf}

\item After obtaining an orthogonal basis for \(V\), we can normalize the
vectors inside to obtain an orthonormal basis for \(V\):

\begin{corollary}
\label{cor:exist-orthonormal-basis}
Let \(V\) be an inner product space where \(\dim(V)\) is finite. Then there
exists an orthonormal basis for \(V\).
\end{corollary}
\begin{pf}
Firstly, there always exists a basis \(S\) for \(V\) (e.g., constructed by the
extension approach). Then, by \Cref{thm:construct-orthog-basis}, we can obtain
an orthogonal basis \(S'\) for \(V\). Then we normalize the vectors in \(S'\)
to get \(\displaystyle \qty{\frac{\vect{v}}{\norm{\vect{v}}}:\vect{v}\in S}\),
which is an orthonormal basis for \(V\).

\begin{note}
Such orthonormal basis is called a \defn{Gram-Schmidt orthonormal basis}, and
the process of obtaining this orthonormal basis as suggested above is said to
be the \defn{Gram-Schmidt process}.
\end{note}
\end{pf}

\item Example: Let \(S=\{\vect{u}_1,\vect{u}_2\}\) where
\(\vect{u}_1=\mqty[1&0&-1]^{T}\) and \(\vect{u}_2=\mqty[2&1&-1]^{T}\), and
consider the inner product space \(V=\spn{S}\) equipped with the standard inner
product. Note that \(S\) is linearly independent, so \(S\) is a basis for
\(V\). However, it is not orthogonal as
\(\inner{\mqty[1&0&-1]^{T}}{\mqty[2&1&-1]^{T}}=3\ne 0\).

To obtain an orthonormal basis for \(V\), we can carry out the Gram-Schmidt
process:
\begin{itemize}
\item \(\displaystyle \vect{v}_1=\vect{u}_1=\mqty[1\\ 0\\ -1].\) 
\item \(\displaystyle
\vect{v}_2=\vect{u}_2-\frac{\inner{\vect{\vect{u}_2}}{\vect{v}_1}}{\norm{\vect{v}_1}^{2}}\vect{v}_1
=\mqty[2\\ 1\\ -1]-\frac{3}{2}\mqty[1\\ 0\\ -1]
=\mqty[1/2\\ 1\\ 1/2].
\)
 \end{itemize}
Hence an orthonormal basis for \(V\) is \{\(\vect{v}_1,\vect{v}_2\)\}. We then
normalize the vectors to get an orthonormal basis for \(V\):
\[
\qty{\frac{1}{\sqrt{2}}\mqty[1\\ 0\\ -1],
\frac{1}{\sqrt{6}/2}\mqty[1/2\\ 1\\ 1/2]
}.
\]
\end{enumerate}
\subsection{Orthogonal Complements}
\begin{enumerate}
\item Next, we will discuss the concept of orthogonal complements, which is
closely related to the idea of \emph{orthogonal projections}.
\item Let \(S\) be a nonempty subset of an inner product space \(V\). Then, the
\defn{orthogonal complement} of \(S\) is
\[
S^{\perp}=\{\vect{v}\in V:\inner{\vect{v}}{\vect{s}}\text{ for any }\vect{s}\in S\},
\]
i.e., the subset of \(V\) containing vectors orthogonal to \emph{all} vectors
in \(S\).

Examples:
\begin{itemize}
\item \(\{\vect{0}\}^{\perp}=V\) by \Cref{prp:in-prod-prop}.
\item \(V^{\perp}=\{\vect{0}\}\) since \(\inner{\vect{v}}{\vect{v}}\ne 0\) for
any \(\vect{v}\ne\vect{0}\), and \(\inner{\vect{0}}{\vect{v}}=0\) for any
\(\vect{v}\in V\) by \Cref{prp:in-prod-prop}.
\item Let \(V=\R^3\) equipped with standard inner product, and let
\(S=\{\vect{e}_2\}\subseteq \R^3\). Then
\(S^{\perp}=\qty{\mqty[x&0&z]^{T}:x,z\in\R}\).


\begin{pf}
``\(\supseteq\)'': Observe that
\(\inner{\mqty[x&0&z]^{T}}{\vect{e}_2}=x(0)+0(1)+z(0)=0\) for any \(x,z\in\R\).

``\(\subseteq\)'': For any \(\vect{v}=\mqty[v_1&v_2&v_3]^{T}\in S^{\perp}\), we
have \(\inner{\vect{v}}{\vect{e}_2}=0\), which implies that \(v_2=0\), thus
\(\vect{v}\in \qty{\mqty[x&0&z]^{T}:x,z\in\R}\).
\end{pf}
\end{itemize}
\item The following provides some properties about orthogonal complements.

\begin{proposition}
\label{prp:orthog-comp-prop}
Let \(V\) be an inner product space and \(S\) be a nonempty subset of \(V\). Then,
\begin{enumerate}
\item \(S^{\perp}=\spn{S}^{\perp}\).
\item \(S^{\perp}\) is a vector subspace of \(V\).
\item \(S\cap S^{\perp}=\{\vect{0}\}\), under the further assumption that \(S\) is a vector subspace of \(V\).
\end{enumerate}
\end{proposition}
\begin{pf}
\begin{enumerate}
\item Recall that
\[
\spn{S}=\qty{\sum_{i=1}^{k}a_i\vect{s}_i:a_1,\dotsc,a_k\in\R, \vect{s}_1,\dotsc,\vect{s}_k\in S, k\in\N}.
\]
``\(\subseteq\)'': For any \(\vect{v}\in S^{\perp}\), we have
\(\inner{\vect{v}}{\vect{s}}=0\) for any \(\vect{s}\in S\). Then, for any
\(k\in\N\), \(\vect{s}_1,\dotsc,\vect{s}_k\in S\), and \(a_1,\dotsc,a_k\in\R\),
we have
\[
\inner{\vect{v}}{\sum_{i=1}^{k}a_i\vect{s}_i}
=\sum_{i=1}^{k}a_i\inner{\vect{v}}{\vect{s}_i}
=0,
\]
thus \(\vect{v}\in\spn{S}^{\perp}\).

``\(\supseteq\)'': For any \(\vect{v}\in\spn{S}^{\perp}\), we have
\(\inner{\vect{v}}{\vect{u}}=0\) for any \(\vect{u}\in\spn{S}\). Note that
\(S\subseteq \spn{S}\), so particularly we have \(\inner{\vect{v}}{\vect{s}}=0\)
for any \(\vect{s}\in\spn{S}\), thus \(\vect{v}\in S^{\perp}\).

\item Fix any \(\vect{u},\vect{v}\in S^{\perp}\) and \(c\in\R\).

\underline{Addition}: For any \(\vect{s}\in S\),
\[
\inner{\vect{u}+\vect{v}}{\vect{s}}
=\inner{\vect{u}}{\vect{s}}
+\inner{\vect{v}}{\vect{s}}
=0+0=0,
\]
thus \(\vect{u}+\vect{v}\in S^{\perp}\).

\underline{Scalar multiplication}: For any \(\vect{s}\in S\),
\[
\inner{c\vect{v}}{\vect{s}}=c\inner{\vect{v}}{\vect{s}}=c(0)=0,
\]
thus \(c\vect{v}\in S^{\perp}\).

\item ``\(\subseteq\)'': For any \(\vect{v}\in S\cap S^{\perp}\), we have
\[
\inner{\underbrace{\vect{v}}_{\in S^{\perp}}}{\underbrace{\vect{v}}_{\in S}}
=0,
\]
which implies that \(\vect{v}=\vect{0}\).

``\(\supseteq\)'': Note that \(\vect{0}\in S^{\perp}\) always since we must
have \(\inner{\vect{0}}{\vect{s}}=0\) for any \(\vect{s}\in S\), by
\Cref{prp:in-prod-prop}. Furthermore, we have \(\vect{0}\in S\) since \(S\) is
a vector subspace, by \Cref{prp:vec-subsp-contain-zero}.
\end{enumerate}
\end{pf}
\item Using the concept of orthogonal complement, we have the following
important result about orthogonal projection.
\begin{theorem}[Orthogonal decomposition theorem]
\label{thm:orthog-decomp}
Let \(V\) be an inner product space and \(W\) be a vector subspace of \(V\).
Then for any \(\vect{v}\in V\), there exist \emph{unique} \(\vect{x}\in W\) and
\(\vect{y}\in W^{\perp}\) such that \(\vect{v}=\vect{x}+\vect{y}\).

The vector \(\vect{x}\) is called the \defn{orthogonal projection} of
\(\vect{v}\), and can be obtained by the formula
\[
\vect{x}=\inner{\vect{v}}{\vect{v}_1}\vect{v}_1+\dotsb+\inner{\vect{v}}{\vect{v}_k}\vect{v}_k.
\]
where \(\{\vect{v}_1,\dotsc,\vect{v}_k\}\) is an orthonormal basis for \(W\).
\end{theorem}
\begin{note}
The formula for \(\vect{x}\) is ``inspired'' by \Cref{thm:orthog-lin-comb}.
(Here \(\vect{v}_1,\dotsc,\vect{v}_k\) all have unit norm.)
\end{note}

\begin{pf}
\underline{Existence}: Let \(\{\vect{v}_1,\dotsc,\vect{v}_k\}\) be an
orthonormal basis for \(W\). First let
\[
\vect{x}=\inner{\vect{v}}{\vect{v}_1}\vect{v}_1+\dotsb+\inner{\vect{v}}{\vect{v}_k}\vect{v}_k,
\]
which belongs to \(W\) since \(\vect{v}_1,\dotsc,\vect{v}_k\in W\).

Then, we let \(\vect{y}=\vect{v}-\vect{x}\). For any \(i=1,\dotsc,k\), we have
\end{pf}
\begin{align*}
\inner{\vect{y}}{\vect{v}_i}&=\inner{\vect{v}-\vect{x}}{\vect{v}_i} \\
&=\inner{\vect{v}}{\vect{v}_i}-\inner{\vect{x}}{\vect{v}_i} \\
&=\inner{\vect{v}}{\vect{v}_i}-\inner{\inner{\vect{v}}{\vect{v}_1}\vect{v}_1+\dotsb+\inner{\vect{v}}{\vect{v}_k}\vect{v}_k}{\vect{v}_i} \\
&=\inner{\vect{v}}{\vect{v}_i}-\inner{\vect{v}}{\vect{v}_i}\underbrace{\inner{\vect{v}_i}{\vect{v}_i}}_{1} &\text{(all other terms vanish due to orthogonality)}\\
&=0.
\end{align*}

\underline{Uniqueness}: Suppose we have
\(\vect{v}=\vect{x}+\vect{y}=\vect{x}'+\vect{y}'\) where
\(\vect{x},\vect{x}'\in W\) and \(\vect{y},\vect{y}'\in W^{\perp}\). Then,
\[
\vect{x}-\vect{x}'=\vect{y}'-\vect{y}.
\]
Since \(\vect{x}-\vect{x}'\in W\) and \(\vect{y}'-\vect{y}\in W^{\perp}\), we
have \(\vect{x}-\vect{x}'=\vect{y}'-\vect{y}\in W\cap W^{\perp}=\{\vect{0}\}\),
by \Cref{prp:orthog-comp-prop}.

Thus, we have \(\vect{x}-\vect{x}'=\vect{y}'-\vect{y}=\vect{0}\), so
\(\vect{x}=\vect{x}'\) and \(\vect{y}=\vect{y}'\).

\item Examples: (Here the inner product spaces are equipped with standard inner
product.)
\begin{itemize}
\item Let \(W=\spn{\{\vect{e}_1,\vect{e}_3\}}\subseteq \R^3\). Then, it can be
checked that \(W^{\perp}=\spn{\vect{e}_2}\). For \(\vect{v}=\mqty[2&3&4]^{T}\),
we can write
\[
\vect{v}=\underbrace{\mqty[2\\ 0\\ 4]}_{\in W}+\underbrace{\mqty[0\\ 3\\ 0]}_{\in W^{\perp}},
\]
where \(\mqty[2&0&4]^{T}\) is the orthogonal projection of \(\vect{v}\).

This expression can be obtained by ``inspection'', or alternatively, by using
the formula
\[
\vect{x}=\inner{\vect{v}}{\vect{e}_1}\vect{e}_1+\inner{\vect{v}}{\vect{e}_3}\vect{e}_3
=\mqty[2\\ 0\\ 4]
\qqtext{and}
\vect{y}=\vect{v}-\vect{x}=\mqty[0\\ 3\\0 ]
\]
since \(\{\vect{e}_1,\vect{e}_3\}\) is an orthonormal basis for \(W\).
\item Let \(W=\spn{\qty{\mqty[1&-2]^{T}}}\). Then, it can be checked that
\(W^{\perp}=\spn{\qty{\mqty[2&1]^{T}}}\). For \(\vect{v}=\mqty[3& 4]^{T}\),
since \(\displaystyle \{\vect{v}_1\}=\qty{\frac{1}{\sqrt{5}}\mqty[1&-2]^{T}}\) is an orthonormal
basis for \(W\), we can use the formula
\[
\vect{x}=\inner{\vect{v}}{\vect{v}_1}\vect{v}_1
=\qty(\frac{3}{\sqrt{5}}-\frac{8}{\sqrt{5}})\cdot \frac{1}{\sqrt{5}}\mqty[1\\ -2]
=\mqty[-1\\ 2]
\qqtext{and}
\vect{y}=\vect{v}-\vect{x}=\mqty[4\\ 2]
\]
to get the decomposition
\[
\vect{v}=\underbrace{\mqty[-1\\ 2]}_{\in W}+\underbrace{\mqty[4\\ 2]}_{\in W^{\perp}}
\]
where \(\mqty[-1&2]^{T}\) is the orthogonal projection of \(\vect{v}\).
\begin{center}
\begin{tikzpicture}
\draw[-Latex, magenta, thick] (0,0) -- (-1,2);
\draw[-Latex, orange, opacity=0.8] (-1,2) -- (3,4);
\draw[-Latex, blue, thick] (0,0) -- (3,4);
\node[blue] () at (2,2) {\(\vect{v}\)};
\node[magenta] () at (-1,1) {\(\vect{x}\)};
\node[orange] () at (1,3.5) {\(\vect{y}\)};
\end{tikzpicture}
\end{center}
\end{itemize}
\end{enumerate}
